% !TeX root = ../main.tex

\xchapter{用于脑卒中快速鉴别的生成式EIT图像重建算法模型}{Model of Generative EIT Reconstruction Algorithm for Fast Detection of Stroke}
传统的EIT图像重建算法通常利用迭代的思想多次求解EIT正问题、逆问题,以不断最小化每一步求解的电压分布和真实电压分布之间的误差。
这种做法通常消耗时间长,成像分辨率低。由于EIT正问题仿真模型完备,且相对精确,而常见的深度学习模型则需要大量的训练数据,因此基于深度学习的EIT图像重建模型大大提高了EIT图像重建算法的重建分辨率,
并且由于训练好的深度学习模型正向推理的过程通常耗时较少,因而能显著提高图像重建算法的时间效率。这也为EIT技术实现床旁实时监测、院前疾病监测提供了技术保障。

目前主流的利用深度学习技术实现的EIT图像重建算法通常以卷积神经网络的各种变体为模型基础,通过添加具有不同功能的模块来优化网络性能。
此类方法通常使用判别式模型来实现。

扩散概率模型,作为生成式模型的一种,近年来在图像生成领域已经取得了显著的成果。而EIT图像重建问题本质上是建立从电压分布到电导率分布的非线性映射,
即可以看作是生成目标为电导率分布,且指导模型生成电导率分布的条件为电压分布的带条件的扩散概率模型。
利用扩散模型实现EIT图像重建算法,由于扩散模型加噪声、去噪声的过程而一定程度上提高了模型抗噪声的能力,并且通过对推断过程的改进,可以有效减少在推理过程中的采样次数,极大程度的缩短了模型正向计算的过程,进而提高了EIT图像重建效率。

本章将重点介绍用于脑卒中快速鉴别的生成式EIT图像重建算法的模型架构。包括模型总体的结构、模型各部分的结构和作用、模型训练集的设计和仿真、训练过程、训练结果以及对模型性能的评估和优化。

\xsection{用于脑卒中快速鉴别的生成式EIT图像重建算法模型架构}{Architecture of the Model of Generative EIT Reconstruction Algorithm for Fast Detection of Stroke}

\xsubsection{模型的总体架构}{The Overall Architecture of The Model}
由于EIT图像重建任务中,不同类型的数据维度不同,同一类型的数据也包含多种表示方式,因此本节将首先建立用于脑卒中快速鉴别的生成式EIT图像重建算法模型,然后再对其各个部分进行逐一阐述。

设待测场$\Omega$所包含的剖分单元个数为$d_e$,边界为$\partial \Omega$。其边界的电极个数为$d_v$。
EIT采集系统的激励模式为邻近激励,单次测量所有激励模式所构成的集合为$S_{stim}$,单次测量电流激励的次数为$n_m$;测量电压的模式为邻近差分,单次激励测量电压的维度为$d_{vm}$。
则单次测量的电压向量维度为$d_m = d_{vm} * n_m$。(关于EIT的有限元模型和测量模式详见)。

则EIT图像重建的电导率分布向量为 $\sigma$,其维度为 $1 \times d_e$;EIT的测量电压可分别表为矩阵形式$v_{mtx} = \{v_{ij}\}, i \in S_{stim}, j =1,2,...d_vm$。其中$v_{ij}$表示以$i$为激励方式的第$j$次测量的结果。
向量形式$v_{vector}$,其维度为$d_m$。EIT重建的图像分辨率为$res_{row} \times res_{column}$。

如无特殊说明,本文所采用的参数为\cref{table:param_stm}中的默认值。
\begin{table}
    \centering
    \label{table:param_stm}
    \caption{参数设置}
    \begin{tblr}{hlines,
        vlines,
        colspec = {X[c] X[c] X[c] X[c]},
    }
 
    参数 & 值(域) & 参数 & 值(域) \\
    \midrule
    $\Omega$ & 圆形场域 $ x \in [-1, 1], y \in[-1, 1]$ & $S_{stim}$ & [i, i+1], i=1:16 \\
    $d_e$ & 1024 & $n_m$ & 16 \\
    $d_v$ & 16 & $d_{vm} $ & 16 \\
    $d_m$ & 256 & $\sigma_i$ & $[1, d_e]$ \\
    $res_{row}$ & 16 & $res_{column}$ & 16 \\

 
    \end{tblr}
\end{table}


如图所示,本文提出的用于脑卒中快速鉴别的生成式EIT图像重建算法模型共包括三个模块,分别是:
\begin{enumerate}
    \item 数据编码块。用于对电压数据、电导率数据进行编码,以输出适应图像重建模块的输入维度。此外,根据测试结果表明,该部分还能提高对应电压-电导率数据之间的相关性,进一步优化网络性能(见)。
    \item 图像重建块。用于拟合EIT图像重建算法,编码后的电导率分布。该模块是本算法的核心部分。
    \item 图像后处理块。用于解码图像重建块的生成结果,将生成结果映射到用于引导图像的电导率分布所适应的维度。
\end{enumerate}
以下将对三个模块分别详细解析。
\xsubsection{数据编码块}{Data Encoding Block}
此模块架构如图所示。由于电压向量的表示方式不同,而神经网络对于输入向量的维度敏感度极高,因此此部分首先将输入的电压向量(或矩阵)映射为一维行向量,其长度为$d_m$。
其中,CVAE、VVAE分别为两个编码器,用于对电导率分布向量和电压分布向量进行编码,其结构分别如\cref{table:VAE_Conductivity}所示。


\begin{table}[H]
    \centering
    \caption{CVAE编码器架构}
    \label{table:VAE_Conductivity}
    \begin{tblr}{
        colspec = {X[c] X[c] X[c] X[c]},
        }
        \toprule
        层数 & 网络类型 & 输入通道数 & 输出通道数 \\
        \midrule
        1 & Conv2d & 1 & 128 \\
        2 & VAE\_ResidualBlock & 128 & 128 \\
        3 & VAE\_ResidualBlock & 128 & 256 \\
        4 & VAE\_ResidualBlock & 256 & 256 \\
        5 & VAE\_ResidualBlock & 256 & 512 \\
        6 & VAE\_ResidualBlock & 512 & 512 \\
        7 & VAE\_SelfAttention & 512 & 512 \\
        7 & GroupNorm &  &  \\
        8 & Conv2d & 512 & 2 \\
        \bottomrule
    \end{tblr}
\end{table}

由于EIT电导率分布数据通常具有二维的特征,因此CVAE利用常见的残差卷积结构作为网络主体,以此来捕获电导率分布的高维特征。

其中,第一层常规的2维卷积,
第二层至第6成均为残差卷积块,结构如\cref{table:VAEResidualBlock}所示。在这里引入残差卷积,可以有效解决整个卷积神经网络的优化问题,
并且可以提高网络的训练速度和收敛性,且能降低过拟合的风险。
在VAE残差卷积块中,用了Group Normalization作为归一化层,能够减少归一化对batch size 的依赖(batch normalization对于batch size的大小非常敏感,其中较小的batch size可能会导致均值和方差不够准确)。
本文所采取的batch size 受计算机性能所限,为32,因此Group Normalization能显著减少网络的内存开销,并且一定程度上提高网络的性能。
此外,该模块中采用了SiLU作为激活函数,相较于一些饱和型激活函数(如Sigmoid和Tanh),可以有效避免在输入较大或较小时导致梯度接近于零,从而避免梯度消失的问题。

\begin{table}
    \centering
    \caption{VAE残差卷积块}
    \label{table:VAEResidualBlock}
    \begin{tblr}{colspec = {X[c] X[c] X[c] X[c]}}
    \toprule
    层数 & 网络类型 & 输入通道数 & 输出通道数 \\
    \midrule
    1 & residule & & \\
    2 & GroupNorm & & \\
    3 & SiLU & &\\
    4 & 2dConv  & in\_channels &   out\_channels \\
    5 & GroupNorm & & \\
    6 & SiLU & & \\
    7 & 2dConv &  out\_channels &  out\_channels  \\
    8 & Conv2d(residule) + outputlayer7 & & \\
    \bottomrule
    \end{tblr}
\end{table}


    其中第七层为SelfAttention层,该层由一个LayerNorm层 和一个SelfAttention模块和组合而成。
其允许模型在处理电导率分布时专注于数据的某些部分(如扰动目标的部分),而忽略其他部分(即背景帧)。
该模块可以使得电导率分布在不同的位置之间保持联系,以实现对扰动部分更好地成像。关于SelfAttention的运算细节见\cref{attention}。

\begin{table}[H]
    \centering
    \caption{VVAE编码器架构}
    \label{table:VAE_Voltage}
    \begin{tblr}{
        colspec = {X[c] X[c] X[c]},
    }
    \toprule
    层数 & 网络类型 \\
    \midrule
    1 & Linear  \\
    2 & residue 1 \\
    3 & LayerNorm \\
    4 & SelfAttention  + residue 1\\
    5 & residue 2 \\
    6 & LayerNorm \\
    7 & Linear \\
    8 & QuickGELU \\
    9 & Linear  + residue 2\\ 
    \bottomrule
    \end{tblr}
\end{table}






\xsubsection{图像重建块}{Image Reconstruction Block}
\xsubsection{图像后处理块}{Postprocessing Block of the Image}


\xsection{用于模型训练和测试的数据集设计与生成}{Design and GeneratioData of Dataset for Model Training and Testing}

如\cref{figure:encoding_1}所示。

\begin{figure}[h]
    \centering
    \includegraphics[width=.5\textwidth]{encoding_1.png}
    \caption{编码后的电导率和真实电导率对比}
    \label{figure:encoding_1}
\end{figure}

\begin{figure}[h]
    \centering
    \includegraphics[width=.5\textwidth]{cond_voltage.png}
    \caption{编码后的电导率和真实电导率对比}
    \label{figure:cond_voltage}
\end{figure}

