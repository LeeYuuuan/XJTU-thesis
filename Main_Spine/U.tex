\begin{figure}[h]
  \includegraphics[width=.7\textwidth]{diffusionforward.png}
  \caption{DDPM加噪过程}
  \label{figure:Diffusionforward}
\end{figure}

\begin{figure}[h]
  \centering
  \includegraphics[width=.7\textwidth]{diffusionreverse.jpg}
  \caption{DDPM去噪过程}
  \label{figure:Diffusionreverse}
\end{figure}



如\cref{figure:ddpm_EIT}所示。

\begin{figure}[h]
    \centering
    \includegraphics[width=.8\textwidth]{ddpm_EIT.JPG}
    \caption{利用扩散概率模型实现EIT图像重建}
    \label{figure:ddpm_EIT}
\end{figure}



模型结构如\cref{figure:DDPMreconstruction}所示。



其通过最小化每一步骤中真实加入的噪声和该网络生成的噪声的均方误差从而训练整个网络使其收敛。
此后,该模型正向推理的过程从一个高斯噪声开始,经过$T:1$时刻通过训练好的去噪网络分别计算出每个时刻需要去掉的噪声,从而获得生成的最终结果。






该方差序列从$\beta_1 = 10^{-4}$开始,到$\beta_T = 0.02$,按照线性插值方式对其插值。其作为每次添加噪声的方差,并通过加权和的方式将噪声加入该时刻代表电导率分布的隐变量中。


\begin{figure}[h]
    \centering
    \includegraphics[width=.9\textwidth]{DDPMreconstruction.PNG}
    \caption{图像重建块模型结构}
    \label{figure:DDPMreconstruction}
\end{figure}


其中,Sampler 为用于采样 $t$时刻加噪声后的数据$x_t$,即Diffusion model的正向扩散过程;
Denoising Net 用于生成每一步需要去掉的噪声分布。该部分需要根据输入的当前时间$t$、引导图像生成的条件向量(本文中为测量的边界电压分布)$v$以及当前时刻生成结果$x_t$,
来预测$t-1$时刻需要减掉的噪声,进一步计算出$x_{t-1}$步的生成结果。不断循环预测的过程,即可回复原始数据$x_0$。
本节将首先介绍整个图像重建块的架构,然后再阐述本文所实现的生成式EIT图像重建算法中生成噪声的网络的架构以及如何利用Cross-Attention机制将电压分布嵌入到网络的输入中。








\xsection{径向基函数神经网络}{}
\label{RBF}




111

111


111
此外,由于EIT图像重建算法固有的病态性,即微小的电压变化就会对重建结果产生巨大的影响,因此本文设计了一个去噪声网络,该网络结构与\cref{RBF}所提出的网络架构基本一致,
不同的是将网络的输出层用一个全连接的网络替代,使得输出结果为一个256维的向量,用于表示该真实电压分布所对应的电压仿真结果,起到了去噪声的效果。
该网络用于拟合真实测量的电压分布到仿真的电压分布的映射,用于对测得的电压分布进行去噪。

具体而言,利用采集到的数据帧进行邻近测量的差分处理后,同时利用同样电导率分布的仿真模型经过EIT正向计算获得电压向量,
利用两个电压向量作为训练数据。其中,仿真的电压数据作为网络的预期输出,而测量所得到的电压数据作为网络的输入数据。
该数据集共包含2000条数据,随后按照训练集-测试集8:2的方式进行划分,由此所得到的去噪网络的

利用该去噪网络,本文对电导率为0.05S/m,底面直径为2.5cm的的扰动目标重新进行了成像,所得到的的结果如\cref{figure:resgood}
\begin{figure}[h]
    \centering
    \includegraphics[width=.5\textwidth]{resgood.png}
    \caption{经过电压去噪声后的重建结果}
    \label{figure:resgood}
\end{figure}

可以看到,经过电压去噪网络后,该算法能有效地重建出小目标的电导率分布。
故在实际的应用中,可以首先针对不同的EIT采集设备设计适用于其的电压去噪声网络,随后去噪后的EIT电压数据进行图像重建任务,
即可最大程度上减少由于测量电压分布于真实电压分布不同而导致的图像重建质量差的问题。
