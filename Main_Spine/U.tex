\begin{figure}[h]
  \includegraphics[width=.7\textwidth]{diffusionforward.png}
  \caption{DDPM加噪过程}
  \label{figure:Diffusionforward}
\end{figure}

\begin{figure}[h]
  \centering
  \includegraphics[width=.7\textwidth]{diffusionreverse.jpg}
  \caption{DDPM去噪过程}
  \label{figure:Diffusionreverse}
\end{figure}



如\cref{figure:ddpm_EIT}所示。

\begin{figure}[h]
    \centering
    \includegraphics[width=.8\textwidth]{ddpm_EIT.JPG}
    \caption{利用扩散概率模型实现EIT图像重建}
    \label{figure:ddpm_EIT}
\end{figure}



模型结构如\cref{figure:DDPMreconstruction}所示。



其通过最小化每一步骤中真实加入的噪声和该网络生成的噪声的均方误差从而训练整个网络使其收敛。
此后,该模型正向推理的过程从一个高斯噪声开始,经过$T:1$时刻通过训练好的去噪网络分别计算出每个时刻需要去掉的噪声,从而获得生成的最终结果。






该方差序列从$\beta_1 = 10^{-4}$开始,到$\beta_T = 0.02$,按照线性插值方式对其插值。其作为每次添加噪声的方差,并通过加权和的方式将噪声加入该时刻代表电导率分布的隐变量中。


\begin{figure}[h]
    \centering
    \includegraphics[width=.9\textwidth]{DDPMreconstruction.PNG}
    \caption{图像重建块模型结构}
    \label{figure:DDPMreconstruction}
\end{figure}


其中,Sampler 为用于采样 $t$时刻加噪声后的数据$x_t$,即Diffusion model的正向扩散过程;
Denoising Net 用于生成每一步需要去掉的噪声分布。该部分需要根据输入的当前时间$t$、引导图像生成的条件向量(本文中为测量的边界电压分布)$v$以及当前时刻生成结果$x_t$,
来预测$t-1$时刻需要减掉的噪声,进一步计算出$x_{t-1}$步的生成结果。不断循环预测的过程,即可回复原始数据$x_0$。
本节将首先介绍整个图像重建块的架构,然后再阐述本文所实现的生成式EIT图像重建算法中生成噪声的网络的架构以及如何利用Cross-Attention机制将电压分布嵌入到网络的输入中。








\xsection{径向基函数神经网络}{}
\label{RBF}




111

111


111
此外,由于EIT图像重建算法固有的病态性,即微小的电压变化就会对重建结果产生巨大的影响,因此本文设计了一个去噪声网络,该网络结构与\cref{RBF}所提出的网络架构基本一致,
不同的是将网络的输出层用一个全连接的网络替代,使得输出结果为一个256维的向量,用于表示该真实电压分布所对应的电压仿真结果,起到了去噪声的效果。
该网络用于拟合真实测量的电压分布到仿真的电压分布的映射,用于对测得的电压分布进行去噪。

具体而言,利用采集到的数据帧进行邻近测量的差分处理后,同时利用同样电导率分布的仿真模型经过EIT正向计算获得电压向量,
利用两个电压向量作为训练数据。其中,仿真的电压数据作为网络的预期输出,而测量所得到的电压数据作为网络的输入数据。
该数据集共包含2000条数据,随后按照训练集-测试集8:2的方式进行划分,由此所得到的去噪网络的

利用该去噪网络,本文对电导率为0.05S/m,底面直径为2.5cm的的扰动目标重新进行了成像,所得到的的结果如\cref{figure:resgood}
\begin{figure}[h]
    \centering
    \includegraphics[width=.5\textwidth]{resgood.png}
    \caption{经过电压去噪声后的重建结果}
    \label{figure:resgood}
\end{figure}

可以看到,经过电压去噪网络后,该算法能有效地重建出小目标的电导率分布。
故在实际的应用中,可以首先针对不同的EIT采集设备设计适用于其的电压去噪声网络,随后去噪后的EIT电压数据进行图像重建任务,
即可最大程度上减少由于测量电压分布于真实电压分布不同而导致的图像重建质量差的问题。





根据以上结果可以看出:

     1)本文所提出的算法可以有效区分不同电导率下的扰动目标,即可以实现对不同类型的卒中进行快速鉴别。
     2)该算法可以有效对场域中不同位置的扰动目标成像。(传统EIT图像重建算法对接近场域中心的扰动目标成像效果差)。
     3)该算法对尺寸较小的扰动目标的成像分辨率不如尺寸较大的扰动目标。


据分析,上述第3条结果可能的原因如下:由于该扰动目标的尺寸较小,同时采集到的数据包含大量噪声,因此该采集结果中信噪比较低,
同时训练集的并没有直接包含该尺寸的扰动目标,故使得该网络无法对该类型的扰动目标准确地提取特征并成像。
此外,但通过结论可以发现,本文所提出的算法对与原先数据分布有很大差异的电压数据其中有关电导率分布值的特征提取能力较强。
即准确地展现出了该扰动目标的电导率分布,这一定程度上体现出该算法的在对扰动目标电导率分布值的拟合的泛化能力较强。
因此可有效地区分出不同电导率分布的扰动目标,进而应用在对于脑卒中快速鉴别的任务中。




\xsubsection{归一化}{Normalization}
在机器学习(尤其是深度学习)任务中,由于不同通道(特征)数据的量纲往往不同,而这种情况通常会影响最终数据分析的结果。
因此,为了消除这种由于量纲不同而导致模型性能下降的可能性,通常会进行数据标准化(standardization),最常用的方法即数据归一化(Normalization)处理。
简而言之,归一化的目的就是使得预处理的数据被限定在一定的范围内(比如$[0,1]$或者$[-1,1]$),从而消除奇异样本数据导致的不良影响。

本文将重点介绍模型中使用到的以下三种归一化:
\begin{enumerate}
  \item 批归一化(Batch Normalization, BN)

  
  通过计算一个batch中每一个通道内的所有数据的每一个特征的均值和方差,然后利用\cref{equation:norm}


其中,$E\left[ x\right], Var\left[ x\right]$分别是上述数据的均值和方差, $\epsilon$为一趋于0的常数,确保分母大于零,
$\gamma, \beta$是可学习的参数,用于对数据进行仿射变换。

该方法可以加快网络的收敛,提高其训练过程的稳定性,还有助于避免产生梯度消失、梯度爆炸的问题。


\begin{figure}[H]
    \centering
    \includegraphics[width=1\textwidth]{SNR30.png}
    \caption{信噪比为30时的重建结果}
    \label{figure:SNR30}
\end{figure}

\begin{figure}[H]
    \centering
    \includegraphics[width=1\textwidth]{SNR20.png}
    \caption{信噪比为20时的重建结果}
    \label{figure:SNR20}
\end{figure}

\begin{figure}[H]
    \centering
    \includegraphics[width=1\textwidth]{SNR15.png}
    \caption{信噪比为15时的重建结果}
    \label{figure:SNR30}
\end{figure}

\begin{figure}[H]
    \centering
    \includegraphics[width=1\textwidth]{SNR10.png}
    \caption{信噪比为10时的重建结果}
    \label{figure:SNR30}
\end{figure}





此模块由一个编码器构成,用于对EIT边界电压数据进行编码。由于电压向量的表示方式不同,而神经网络对于输入向量的维度敏感度极高,因此此部分首先将输入的电压向量(或矩阵)映射为一维行向量,其长度为$d_m$。
由于VEncoder用于编码边界电压数据,因此要求此编码器能实现对电压分布的深层次特征进行提取,从而更好地捕获其特征与电导率分布之间的关系。
基于上述目的,本文采用残差卷积模块为主体设计了一个编码器,其具体结构如\cref{table:VAE_Voltage}所示。
\begin{table}[H]
    \centering
    \caption{VEncoder编码器架构}
    \label{table:VAE_Voltage}
    \begin{tblr}{
        colspec = {X[c] X[c] X[c]},
    }
    \toprule
    层数 & 输入 & 网络类型  & 输出\\
    \midrule
    1 & $input$ & Linear $\times 5$ & $v_1=\text{Linear}(input)$, $residue_1=input$  \\
    2 & $v_1$ & LayerNorm & $v_2=\text{LayerNorm}(v_1)$ \\
    3 & $v_2, residue_1$ & SelfAttention & $v_3 = \text{SelfAttention}(v_2) + residue_1, residue_2 = v_2$\\
    4 & $v_3$ & Linear $\times 5$& $v_4 = \text{Linear}(v_3)$ \\
    5 & $v_4$ & QuickGELU & $v_5 = \text{QuickGELU}(v_4)$ \\
    6 & $v_5, residue_2$ & Linear & $v_6 = \text{Linear}(v_5) + residue_2$\\
    7 & $output$ & Linear & $output = \text{Linear}(v_6)$\\
    \bottomrule
    \end{tblr}
\end{table}

\cref{table:VAE_Voltage}中,每一行依次代表该网络中的每一层;第一列(层数)代表当前的层数,第二列(输入)和第四列(输出)
分别表示改层输入的变量和输出的变量,第三列(网络类型)代表主要构成该层的网络类型。
其中,第1、4、6、7为多层感知机,而$residue_1, residue_2$分别为残差项,用于实现残差结构,进一步加快网络收敛并防止梯度消失。



\cref{figure:reconsresult}。
\begin{figure}[h]
    \centering
    \includegraphics[width=.5\textwidth]{reconsresult.jpg}
    \caption{重建结果}
    \label{figure:reconsresult}
\end{figure}

其中第一列是原始的电导率分布图像,第二列是利用CNN重建后的结果,第三列则是本文所提出的EC-Diffusion重建算法所得到的结果。
可以看出,该算法在对图像背景电导率以及边缘电导率的重建效果均明显优于利用CNN重建的图像。

随后本文将该模型的最优参数与其他模型的结果作对比,如\cref{table:EvaluationNice}




\begin{table}[H]
  
    
    \caption{各模型最优参数对比}
    \begin{tblr}{
        colspec = {X[c] X[c] X[c]},
    }
    \toprule
    算法类型 & CC & RE($\times 10^{-5}$) \\
    \midrule
    EC-Diffusion & 0.9780 & 2.1073 \\
    CNN+RBFNN(Adam) & 0.9665 & 4.7563 \\
    CNN(Adam) & 0.3775 & 13.7632 \\
    \bottomrule
    \end{tblr}
    \label{table:EvaluationNice}
\end{table}

其中,EC-Diffusion为本文所提出的生成式EIT图像重建算法,CNN+RBFNN(Adam)和CNN(Adam)为\cite{RBFEIT}所提出的重建算法,
这里均已选择每种模型可能的最优参数来做比较。
可以看出,本文所提出的EC-Diffusion图像重建算法各项指标均高于其他算法。
