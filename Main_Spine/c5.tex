% !TeX root = ../main.tex

\xchapter{总结与展望}{Conclusions and Prospect}

\xsection{结论}{Conclusions}

EIT技术由于其安全、实时、低成本等特点,近年来,越来越多的学者尝试将其各个医疗领域中,其中就包括对于脑卒中的鉴别任务中。而传统的EIT技术由于其图像重建算法固有的病态性,这就导致其重建分辨率较低,重建速度较慢。为了解决该问题,学多学者提出了改进的EIT图像重建算法。其中,部分学者在传统算法的基础上优化了其重建步骤,从而一定程度上提高了重建分辨率。而近年来随着深度学习技术在各个领域得到了广泛的应用,越来越多的学者也开始采用基于深度学习的方法实现高质量的EIT图像重建算法,并一定程度上推进了EIT技术的广泛使用。
 
随着生成式对抗网络、扩散模型等深度生成式模型的提出和广泛应用,生成式模型在图像重建,图像生成等领域取得了前所未有的成果。而目前大多数基于深度学习的EIT图像重建算法都以判别式模型为主要架构。相较于判别式模型,生成式模型往往具有对原始数据分布更强大的拟合能力以及更强的泛化性能。并且目前大部分基于深度学习的EIT图像重建工作均停留在仿真阶段,因此本文将针对脑卒中快速鉴别这一具体任务,实现了一个生成式的EIT图像重建算法。主要采用仿真和开展物理实验的方式,证明了该算法的可行性,并对该算法区分不同电导率分布的扰动目标的性能开展研究。其主要研究内容如下:
 
首先,研究了EIT、深度学习理论。具体而言,
1)研究了EIT技术的原理和优势、EIT图像重建算法的实现过程以及其不足。
2)目前基于深度学习的EIT图像重建算法的优势以及不足。
3)研究了生成式模型(尤其是VAE和扩散模型)以及将生成式模型的架构、原理以及其应用在EIT图像重建算法上的可行性、方案以及优势,从而为实现生成式EIT图像重建算法提供了理论依据和技术支撑。
并得出以下结论:
1)EIT技术通过向人体内注入安全电流随后测量体表的电压分布,进而获得人体内部的电导率分布图像。该技术由于其无辐射、可实时监测等优势,在临床实践中有巨大的应用前景。
2)由于神经网络模型往往具有对非线性映射强大的拟合能力,基于深度学习的EIT图像重建算法具有潜在的应用前景。然而当前主流的基于深度学习的EIT图像重建算法均采用CNN等判别式结构实现,相较于生成式模型此类算法对于原始分布的拟合能力较弱。
3)生成式模型由于其强大的泛化能力以及其对原始数据分布强大的拟合能力因而可应用在EIT图像重建任务中。

其二,研究了生成式EIT图像重建算法的设计与实现方式。
根据上述理论基础,本文提出了一个基于编码器-解码器以及图像重建块的三阶段EIT图像重建算法。
该算法首先利用编码器对EIT数据进行编码,随后利用改进的扩散概率模型以及编码后的数据重建出编码后的图像,
最后利用解码器对该数据进行解码。经仿真数据训练并验证,该算法能重建出高质量的电导率分布图像,并且能大大提高其效率,
为后续的物理实验提供了算法的基础。
 
最后,本文针对脑卒中快速鉴别这一任务设计出物理实验,利用琼脂块分别制作出代表缺血性卒中和出血性卒中的扰动目标,
并利用搭建好的EIT采集系统获取了不同电导率分布的扰动目标的边界电压数据。
然后利用本文所提出的生成式EIT图像重建算法重建出不同类型的扰动目标的EIT图像。
实验结果表明:

1)该算法能有效的区分不同电导率分布的扰动目标。
 
2)由于电压编码器中全连接网络的大量使用而对于整个模型性能产生了负面影响。
故本利用RBFNN和CNN设计了一个全新的电压编码器,然后设计实验验证了该编码器的时间、空间性能。结果表明,该编码器显著的提高了算法的时间、空间性能。

此外,本文还研究了电压编码器和电导率编码器的性能。首先计算了网络中部分中间变量的相关性。根据结果得出,本文所提出的两个编码器显著的提高了编码后的EIT电压-电导率分布之间的相关性。
随后将该编码器所得到的中间结果可视化。通过编码图可以发现,编码器显著地提取到了电压分布中有关电导率分布的深层次特征。
由此得出,该编码器-解码器结构对于EIT图像重建算法的性能有显著的提高。

根据以上所有结论可得,本文所提出的用于脑卒中快速鉴别的生成式EIT图像重建算法能够快速、准确地分辨出不同类型的卒中所对应的电导率分布,
为后续的临床试验提供了算法基础,并有望应用院前的脑卒中的快速鉴别任务中。
此外,该算法也验证了生成式模型对于EIT图像重建算法的可行性,为后续利用生成式模型实现EIT图像重建算法奠定了基础,
并且为其他特定任务下的EIT图像重建任务提供了潜在的研究方向。
 
\xsection{展望}{Prospect}
 
本文所提出的基于生成式的EIT图像重建算法是把生成式模型应用在EIT图像重建算法中的一种尝试,仍有许多不足之处。
 
首先,神经网络的不可解释行仍是该类算法应用在医学检验技术技术中的一大阻碍。故如何在保证模型可解释行的前提下提高EIT图像重建算法的质量,是后续研究的重点目标。
针对该问题,可以将深度学习模型应用于传统算法的求解的某一个步骤中。该做法通过深度学习模型有效地提高了EIT图像重建的质量,并能降低时间开销。
与此同时,由于传统算法具有完善的数学物理模型,因此该做法也能一定程度上保证算法的可解释性。
 
第二,由于物理模型所产生的电压数据与真实的电压分布有一定差异,根据本文的结论可以看出,该算法在仿真数据上的表现明显优于物理模型。故如何设计实验消除这种电压分布之间的差异,是后续研究应当重点关注的内容之一。
针对该问题,本文认为应当重点研究仿真数据中电压分布于物理模型中电压分布的差异。具体而言,可以利用深度学习模型拟合两者电压分布之间的映射,从而对物理模型实验数据去噪。