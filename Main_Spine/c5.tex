% !TeX root = ../main.tex

\xchapter{总结与展望}{Conclusions and Prospect}

\xsection{结论}{Conclusions}

EIT技术由于其安全、实时、低成本等特点,近年来,越来越多的学者尝试将其各个医疗领域中,其中就包括对于脑卒中的鉴别任务中。而传统的EIT技术由于其图像重建算法固有的病态性,这就导致其重建分辨率较低,重建速度较慢。为了解决该问题,学多学者提出了改进的EIT图像重建算法。其中,部分学者在传统算法的基础上优化了其重建步骤,从而一定程度上提高了重建分辨率。而近年来随着深度学习技术在各个领域得到了广泛的应用,越来越多的学者也开始采用基于深度学习的方法实现高质量的EIT图像重建算法,并一定程度上推进了EIT技术的广泛使用。
 
随着生成式对抗网络、扩散模型等深度生成式模型的提出和广泛应用,生成式模型在图像重建,图像生成等领域取得了前所未有的成果。而目前大多数基于深度学习的EIT图像重建算法都以判别式模型为主要架构。相较于判别式模型,生成式模型往往具有对原始数据分布更强大的拟合能力以及更强的泛化性能。并且目前大部分基于深度学习的EIT图像重建工作均停留在仿真阶段,因此本文将针对脑卒中快速鉴别这一具体任务,实现了一个生成式的EIT图像重建算法。主要采用仿真和开展物理实验的方式,证明了该算法的可行性,并对该算法区分不同电导率分布的扰动目标的性能开展研究。其主要研究内容如下:
 
首先对EIT技术的原理、EIT图像重建算法的实现过程进行了研究。随后又对目前基于深度学习的EIT图像重建算法进行了研究。最后对生成式模型以及将生成式模型的架构、原理以及其应用在EIT图像重建算法上的可行性与方案进行了研究。EIT技术通过向人体内注入安全电流随后测量体表的电压分布,进而获得人体内部的电导率分布图像。该技术由于其无辐射、可实时监测等优势,在临床实践中有巨大的应用前景。由于神经网络模型往往具有对非线性映射强大的拟合能力,基于深度学习的EIT图像重建算法具有潜在的应用前景。而生成式模型由于其强大的泛化能力以及其对原始数据分布强大的拟合能力。本文通过对生成式模型尤其是VAE和扩散模型的深入研究从而为实现生成式EIT图像重建算法提供了理论依据和技术支撑。
 
二是对生成式EIT图像重建算法的研究。根据上述理论基础,本文提出了一个基于编码器-解码器以及图像重建块的三阶段EIT图像重建算法。该算法首先利用编码器对EIT数据进行编码,随后利用改进的扩散概率模型以及编码后的数据重建出编码后的图像,最后利用解码器对该数据进行解码。经仿真数据训练并验证,该算法能重建出高质量的电导率分布图像,并且能大大提高其效率,为后续的物理实验提供了算法的基础。
 
随后本文针对脑卒中快速鉴别这一任务设计出物理实验,利用穷之快分别制作出代表缺血性卒中和出血性卒中的扰动目标,并利用搭建好的EIT采集系统获取了不同电导率分布的扰动目标的边界电压数据。然后利用本文所提出的生成式EIT图像重建算法重建出不同类型的扰动目标的EIT图像。实验结果表明,该算法能有效的区分不同电导率分布的扰动目标。
 
此外,本文还对实验结果进行了分析,由于物理实验中噪声的存在导致其重建性能下降,本文还设计了一个去噪声的网络用于去除EIT采集设备所采集到的电压数据的噪声。经验证,该网络能够有效地将采集到的电压分布映射到仿真所得的理论电压分布中,一定程度上避免了噪声对于重建结果的影响。
 
本文所提出的用于脑卒中快速鉴别的生成式EIT图像重建算法能够快速、准确地分辨出不同类型的卒中所对应的电导率分布,,为后续的临床试验提供了算法基础,并有望应用院前的脑卒中的快速鉴别任务中。此外,该算法也验证了生成式模型对于EIT图像重建算法的可行性,为后续利用生成式模型实现EIT图像重建算法奠定了基础,并且为其他特定任务下的EIT图像重建任务提供了潜在的研究方向。
 
\xsection{展望}{Prospect}
 
本文所提出的基于生成式的EIT图像重建算法是把生成式模型应用在EIT图像重建算法中的一种尝试,仍有许多不足之处。
 
首先,神经网络的不可解释行仍是该类算法应用在医学检验技术技术中的一大阻碍。故如何在保证模型可解释行的前提下提高EIT图像重建算法的质量,是后续研究的重点目标。
 
第二,由于物理模型所产生的电压数据与真实的电压分布有一定差异,根据本文的结论可以看出,该算法在仿真数据上的表现明显优于物理模型。故如何设计实验消除这种电压分布之间的差异,是后续研究应当重点关注的内容之一。