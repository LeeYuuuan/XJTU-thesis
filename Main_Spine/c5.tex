% !TeX root = ../main.tex

\xchapter{总结与展望}{Conclusions and Prospect}

\xsection{结论}{Conclusions}

EIT技术由其安全、实时、低成本等特点,近年来被
越来越多的学者广泛应用于医疗领域中,其中就包括对于脑卒中类型的鉴别任务中。
传统的EIT技术由于其逆问题固有的病态性,这就导致其重建分辨率较低,
重建速度较慢。为了解决该问题,许多学者提出了改进的EIT图像重建算法。
其中,部分学者在传统算法的基础上优化了其重建步骤,从而一定程度上提高了重建分辨率。
近年来随着深度学习技术在各个领域表现出优秀的性能,
越来越多的学者也开始采用基于深度学习的方法实现高质量的EIT图像重建算法,此类方法一定程度上推进了EIT技术更广泛的应用。
 
随着生成式对抗网络、扩散模型等深度生成式模型的提出,
其在图像重建、图像生成等领域取得了前所未有的成果。
目前大多数基于深度学习的EIT图像重建算法都以判别式模型为主要架构。
相较于判别式模型,生成式模型往往具有对原始数据分布更强大的拟合能力以及
泛化性能。并且目前大部分基于深度学习的EIT图像重建工作均停留在仿真阶段,
因此本文将针对脑卒中快速鉴别这一具体任务,设计了一个生成式的EIT图像重建算法——EC-Diffusion 算法。
随后设计了一种EIT物理实验模型数据生成策略,用于模拟不同类型脑卒中患者的EIT数据。最后根据分析了该算法的性能,并指出其有效性和先进性。其主要研究内容如下:
 
本文的第二章主要研究了EIT技术及深度学习理论。具体而言,
1)研究了EIT技术的原理和优势、EIT图像重建算法的实现过程以及其不足。
2)目前基于深度学习的EIT图像重建算法的优势以及不足。
3)研究了生成式模型(尤其是VAE和扩散模型)以及生成式模型的架构、原理以及其应用在EIT图像重建算法上的可行性、方案以及优势,从而为实现生成式EIT图像重建算法提供了理论依据和技术支撑。
并得出以下结论:
1)EIT技术通过向人体内注入安全电流随后测量体表的电压分布,进而获得人体内部的电导率分布图像。该技术由于其无辐射、可实时监测等优势,在临床实践中有巨大的应用前景。
2)由于神经网络模型往往具有对非线性映射强大的拟合能力,基于深度学习的EIT图像重建算法具有潜在的应用前景。然而当前主流的基于深度学习的EIT图像重建算法均采用CNN等判别式结构实现,相较于生成式模型此类算法对于原始分布的拟合能力较弱。
3)生成式模型由于其强大的泛化能力以及其对原始数据分布强大的拟合能力因而可应用在EIT图像重建任务中。

第三章主要研究了生成式EIT图像重建算法的设计与实现方式。
根据上述理论基础,本文提出了一个基于编码器-解码器以及图像重建块的多模块EIT图像重建算法。并将其命名为EC-Diffusion算法。
其中,该算法首先利用编码器对EIT数据进行编码,输出编码后的电压分布数据。随后利用Cross Attention 机制实设计并实现了一个改进的扩散概率模型,并将其作为EIT图像重建算法的主要部分。
最后实现了一个用于对上述编码过程解码的解码器。经仿真数据训练并验证,该算法能重建出高质量的电导率分布图像,
为后续将EC-Diffusion算法应用在真实的物理模型上提供了算法的基础。
 
第四章针对脑卒中快速鉴别这一任务设计了一种EIT物理模型实验数据生成策略,利用琼脂块分别制作出代表缺血性卒中和出血性卒中的扰动目标,
并利用搭建好的EIT采集系统获取了不同电导率分布的扰动目标的边界电压数据。
然后利用本文所提出的生成式EIT图像重建算法对上述结果重建,并将其与目前先进的方法作对比,以说明其先进性。此外,本章深入研究了EC-Diffusion模型的抗噪性能以及编码器-解码器的性能,进一步说明各个模块的必要性和先进性。
实验结果表明:
1)该算法能够重建出代表不同类型脑卒中的的扰动目标的EIT电导率分布图像,且重建图像质量优于目前CNN-Based图像重建算法。
2)该模型对噪声具有很强的鲁棒性,这也是该模型能够应用在真实的EIT物理模型中所必要的条件。
3)该模型的编码器能显著提高EIT数据中电压分布和电导率分布之间的相关性,并显著提高了EIT图像重建算法的性能。


根据以上所有结论可得,本文所提出的用于脑卒中快速鉴别的生成式EIT图像重建算法能够快速、准确地分辨出不同类型的卒中所对应的电导率分布,
为后续的临床实验提供了算法基础,并有望应用院前的脑卒中的快速鉴别任务中。
此外,该算法也验证了生成式模型对于EIT图像重建算法的可行性,为后续利用生成式模型实现EIT图像重建算法奠定了基础,
并且为其他特定任务下的EIT图像重建任务提供了潜在的研究方向。
 
\xsection{展望}{Prospect}
 
本文所提出的基于生成式的EIT图像重建算法是把生成式模型应用在EIT图像重建算法中的一种尝试,仍有许多不足之处。
 
首先,神经网络的不可解释行仍是该类算法应用在医学检验技术技术中的一大阻碍。故如何在保证模型可解释行的前提下提高EIT图像重建算法的质量,是后续研究的重点目标。
针对该问题,可以将深度学习模型应用于传统算法的求解的某一个步骤中。该做法通过深度学习模型有效地提高了EIT图像重建的质量,并能降低时间开销。
与此同时,由于传统算法具有完善的数学物理模型,因此该做法也能一定程度上保证算法的可解释性。
 
第二,由于物理模型所产生的电压数据与真实的电压分布有一定差异,根据本文的结论可以看出,该算法在仿真数据上的表现明显优于物理模型。故如何设计实验消除这种电压分布之间的差异,是后续研究应当重点关注的内容之一。
针对该问题,本文认为应当重点研究仿真数据中电压分布于物理模型中电压分布的差异。具体而言,可以利用深度学习模型拟合两者电压分布之间的映射,从而对物理模型实验数据去噪。