% !TeX root = ../main.tex

\xchapter{用于脑卒中快速鉴别的EIT实验数据生成及算法性能分析}{Data Generation of EIT Physical Model for Fast Detection of Stroke and Analysis of Algorithm Performance}

上一章通过EIDORS生成的仿真数据对本文提出的EC-Diffusion EIT图像重建算法中的可学习参数进行了训练和评估。结果表明,该算法能有效重建出高质量的EIT图像。
然而,由于真实的EIT采集过程通常具有不确定性,即包含各种噪声。故绝大多数EIT图像重建算法通过实际的物理模型所获得的重建图像质量通常低于其在仿真模型上的结果。

针对EIT图像重建算法的上述特点,本章首先针对脑卒中快速鉴别任务设计了EIT物理模型实验,随后提出了一种EIT数据生成策略,用于生成模拟脑卒中患者的EIT数据。最后利用仿真数据以及物理模型实验数据分析EC-Diffusion算法的性能,并与同类先进算法作对比,验证了其有效性和先进性。
并为后续展开临床实验奠定基础。

\xsection{EIT物理模型实验设计}{Design of Experiment of EIT Physical Model}
由于EIT物理模型实验的设计方式种类繁多,其实验目的也各不相同,因此本节首先对脑卒中快速鉴别任务做出定量分析,随后针对该任务设计了EIT物理模型实验,
为后续生成用于模拟脑卒中患者的EIT数据提供实验基础。

\xsubsection{脑卒中快速鉴别任务分析}{Analysis of Fast Detection of Stroke}

\subsubsection{脑卒中类型鉴别}
脑卒中由其致病原因不同可分为缺血性脑卒中和出血性脑卒中。出血性卒中由于大量出血而导致其病灶区域局部含血量升高,反之缺血性卒中则由于病灶区域缺血而导致其含血量下降,
由于血液中的主要成分是水,因此以上两种类型的卒中的含水量也具有显著差异,从而会导致卒中的病灶区域的电导率具有显著差异(含水量会显著影响其电导率分布)。

故可假设病灶区域健康时的电导率分布为$\sigma_0$,由于出血而导致的卒中的病灶区域的电导率分布为$\sigma_H$,由缺血而导致的卒中的病灶区域的电导率分布为$\sigma_I$,
则若能够检测卒中区域的电导率提升或降低(即能准确区分出$\sigma_H - \sigma_0$和$\sigma_I - \sigma_0$),即可实现对脑卒中类型的鉴别。

\subsubsection{脑卒中快速鉴别的时间窗口}
由于脑卒中的救治存在严格的时间窗口,因此尽早对脑卒中类型作出诊断并展开救治,能够有效的降低脑卒中的致死致残率。
目前颅脑CT是诊断出血性脑卒中的“金标准”,如\cref{figure:cuzhongleixing}。该技术对于早期脑梗死灶不敏感,而MRI则可以发现早期的缺血性脑卒中。
此类设备庞大,操作复杂,不便于携带,故无法在脑卒中患者到达医院之前判断其类型。因此若能实现院前的卒中类型鉴别,即可被认为是“快速鉴别”。
\begin{figure}[h]
    \centering
    \includegraphics[width=.6\textwidth]{cuzhongleixing.JPG}
    \caption{CT所得到的脑卒中成像结果(左图为缺血性卒中,右图为出血性卒中)\cite{5335289}}
    \label{figure:cuzhongleixing}
\end{figure}

\xsubsection{EIT成像目标的设计与制作}{Design and Manufacture of EIT Imaging Target}
\label{ImageingTarget}
EIT成像目标包括背景帧以及扰动目标,本节将首先介绍扰动目标的制作流程,随后将阐述背景帧的制作流程。

由于脑卒中可被分为缺血性卒中和出血性卒中,缺血性卒中由于其病灶区域血液(水)含量较少,因此其对应的电导率较低。
而出血性卒中由于其病灶区域大量出血,因此其对应的电导率通常较高。

琼脂是亲水性胶体,在一定温度条件下,向琼脂溶液中加入一定量的电介质即可使其具有特定的电导率。
因此本实验将用琼脂和氯化钠以及水制作出具有不同电导率的琼脂快,来代表不同类型卒中的病灶区域。
本实验将采用2个不同的电导率值作为扰动目标的电导率分布,其分别为0.5S/m 和 0.05S/m。

根据多次实验得出,在每100g水,1.5g琼脂粉的条件下,其琼脂块的电导率$\sigma$(S/m)与氯化钠的含量$x$(g)可用\cref{equation:NACL}近似拟合。
本实验制作出不同电导率分布的琼脂块后还利用阻抗分析仪对其进行了验证,以保证扰动目标电导率分布的准确性。
\begin{equation}
    \centering
    \label{equation:NACL}
    \sigma = 2.008x + 0.0207
\end{equation}


此外,本实验用于盛放琼脂并使其凝固的容器为底面直径不同的陶瓷容器,其底面直径分别为5cm,4cm和2.5cm。
故本文所制作的扰动目标的详细参数如\cref{table:target}所示。
\begin{table}[H]
    \centering
    \caption{扰动目标的参数}
    \begin{tblr}{colspec = {X[c] X[c] X[c] X[c]}}
        \toprule
        编号 & 电导率(S/m) & 底面直径(cm) & 每100g水所需的氯化钠含量(g)\\
        \midrule
        1 & 0.05 & 5 & 0.02 \\
        2 & 0.05 & 4 & 0.02 \\
        3 & 0.05 & 2.5 & 0.02 \\
        4 & 0.5 & 5 & 0.23 \\
        5 & 0.5 & 4 & 0.23 \\
        6 & 0.5 & 2.5 & 0.23 \\
        \bottomrule
    \end{tblr}
    \label{table:target}
\end{table}


其具体步制作骤如\cref{figure:experiment},即:

1) 用电子秤称取1.5g的琼脂粉。

2) 将称好的琼脂粉倒入烧杯中。

3) 像烧杯中加入100ml的蒸馏水,并将其与琼脂粉充分混合。

4) 再利用电子秤称取一定量的氯化钠粉末。

5) 将称取的氯化钠粉末倒入琼脂溶液中。

6) 利用高温加热炉,将上述混合溶液加热至沸腾,并在加热的过程中充分搅拌使其溶解(根据实测在常压下需要保持270摄氏度左右)。

7) 将充分溶解的溶液倒入尺寸不同的陶瓷容器中,静置2小时待其凝固。

8) 将凝固的琼脂块取出备用。

以上过程为EIT物理模型中扰动目标的制作方式,上述两种扰动目标将分别代表两类卒中的病灶区域。本实验将用硫酸钙溶液来模拟颅内的正常脑组织,作为病灶区域的背景帧溶液。
由于电解质的溶解率受到温度、压强、震动等外在因素影响显著,而溶液的电导率对电解质含量非常敏感,因此本实验将利用饱和硫酸钙溶液作为背景帧,以将制作过程中产生的误差降到最低,从而使得仿真过程中使用到的背景电导率与物理实验的背景电导率在最大程度上保持一致。
具体而言,在常温(20-24摄氏度)常压(一个标准大气压),将硫酸钙粉末不断加入到大量纯净水中,静置并等待其完全溶解。重复上述步骤,直到硫酸钙不再溶解,静置24小时后获得常温常压下稳定的饱和硫酸钙溶液。
将溶液倒入EIT采集系统的丙烯酸玻璃器皿中,静置24小时待其稳定后方可使用。
此外,由于扰动目标与背景溶液接触时,不可避免的会发生离子交换,从而产生误差。因此,为了将这部分误差降至最低,本实验将制作3批作为背景电导率的饱和硫酸钙溶液,每使用6个琼脂块进行成像后将更换背景溶液,从而尽可能得减少背景溶液的电导率的变化。






\begin{figure}[H]
    \centering
    \includegraphics[width=.7\textwidth]{experiment.png}
    \caption{EIT扰动目标制作流程}
    \label{figure:experiment}
\end{figure}

随后利用阻抗分析仪验证了EIT扰动目标电导率的准确性。此外,本实验中,为了更好的区分不同电导率的扰动目标,用黄色琼脂块表示电导率为0.5S/m的扰动目标
,而用蓝色琼脂块表示电导率为0.05S/m的扰动目标。如\cref{figure:QiongZhiKuai}所示。


\begin{figure}[h]
    \centering

    \includegraphics[width=.7\textwidth]{QiongZhiKuai.png}
    \caption{琼脂块制作示意图}
    \label{figure:QiongZhiKuai}
\end{figure}



\xsubsection{EIT采集系统设计与搭建}{Design and Construction of EIT Measurment System}


EIT采集系统通常包括激励测量硬件系统、电极与导联系统,以及用于控制激励测量过程、分析和保存数据的软件系统。
本实验采用标准的16电极圆形物理模型容器(16 phantom tank)作为待测场。
该水槽的外直径为20cm,内直径为19.4cm,其高度为15cm,材质为丙烯酸玻璃。
该模型共包含16个位于同一平面上的电极,每个电极之间的距离相等。
最后将扰动目标放入其待测场内,即可制作完成用于测量的电导率分布未知场。如\cref{figure:raodongmubiao}所示。
\begin{figure}[H]
    \centering
    \includegraphics[width=.5\textwidth]{raodongmubiao.jpg}
    \caption{待测场图示}
    \label{figure:raodongmubiao}
\end{figure}
测量仪器采用Sciospec Scientific Instruments公司所生产的16通道EIT系统。本实验所采用的电极激励模式为邻近激励,即单次测量电流的输入和输出分别为1号电极和2号电极,2号电极和3号电极,以此类推,共16次激励。
每次电流激励下该系统将分别测量16个电极的电位,故单次测量所获得的电位向量为256维。另外,本实验所测得的数据是在单频激励下所得到的,其激励电流的频率为50khz,
激励电流的强度0.0005A。每秒采样的次数为1帧(即该系统理论上每秒向PC输出1组测量电位)。




\xsection{EIT物理模型实验数据生成}{Generation of EIT Physical Model Experiment}
上一节针对脑卒中快速鉴别任务,设计了一个EIT物理模型实验。
本节将利用上一节设计的EIT物理模型实验,生成一组EIT物理模型实验数据,用于模拟脑卒中患者的EIT数据。


\xsubsection{EIT物理模型实验数据采集}{Data Acquisition of EIT Physical Model Experiment}

本节将具体地阐述EIT物理实验数据采集过程,其共分为两个主要部分。第一部分将利用电导率为0.05S/m的琼脂块作为扰动目标,(即\cref{ImageingTarget}中编号为$1-3$的琼脂块。)分别测量其在水槽模型中若干个不同位置下的电位数据。
第二部分则是利用电导率为0.5S/m的琼脂块作为扰动目标(\cref{ImageingTarget}中编号为4-6的琼脂块),分别测量其在若干位置下的电位数据。

具体开展流程如下:

首先按照\cref{ImageingTarget}中的步骤和具体参数制作电导率为0.05S/m的琼脂块3个。
随后进行第一次实验数据采集。
为防止琼脂块在空气中水分子的流失导致其形变,进而影响其放置在水槽中的稳定性,第二次测量采集需要重新制作电导率为0.5S/m的琼脂块3个,制作好后立即展开第二部分实验。

第一部分其具体流程为:

     1)将静置好的EIT物理实验水槽模型连接EIT采集系统,并将EIT采集系统连接至PC,随后通过其串口设置好电流强度、激励频率、激励模式等参数(见上一节),打开EIT采集系统。

     2)静置1分钟后再等待20秒,记录20帧背景帧的电位数据的编号信息。

     3)向硫酸钙溶液中位置1处放置1号扰动目标,静置1分钟后记录采集到的20帧电位数据的编号。

     4)将1号扰动目标移动至位置2,静置1分钟后记录采集到的20帧电位数据的编号。

     5)将1号扰动目标移动至位置3,重复上述步骤。

     6)将1号扰动目标取出,将2号、3号扰动目标按照上述流程重复静置、记录的步骤。

     7)将扰动目标取出,关闭EIT设备。


第二部分对扰动目标为0.5S/m的三个琼脂块开展与第一部分流程相同的实验。至此已获得初步的物理实验数据待后续使用。

经上述采集流程后,共获得50组EIT电位数据以及其对应的电导率分布(通过相机拍摄得到的RGB图像)。

\xsubsection{EIT物理实验数据集生成}{Generation of EIT Physical Experiment Data}
\label{geneD}
本文对上述测得的电位数据进行了分析。由于EIT采集设备所得到的电位向量为测量电极相对于0电位点的电位差,该分布与本文所提出的差分EIT重建算法重建目标不同,故还需要对采集到的数据进行处理。

具体操作如下:
该设备采集到的EIT数据共包含16次相邻激励的每个电极所采集到的电位值,该值为一个复数,包含了实部和虚部。而EIT图像重建任务的重建目标(电导率分布)仅和电位的实部相关,因此首先去掉所有电位数据的虚部。
第二,通常EIT所采用的电压信号为两个电极之间的电位差,而本文所采用的电压测量模式为邻近测量,其测量值与仿真数据的电压分布有如 \cref{table:measU} 的映射方式。
\begin{table}[H]
    \centering
    \caption{测量电位的转换方式}
    \begin{tblr}{colspec = {X[c] X[c] X[c] }}
        \toprule
        EIT仿真模型中单次测量所得的向量的下标  & EIT采集系统所得到的与该电压值相关的测量值下标 & 利用测量值获得对应的仿真数据方式\\
        \midrule
        0 & 0,1 & $v_0 - v_1$ \\
        1 & 1,2 & $v_1 - v_2$ \\
        2 & 2,3 & $v_2 - v_3$ \\
        3 & 3,4 & $v_3 - v_4$ \\
        4 & 4,5 & $v_4 - v_5$ \\
        5 & 5,6 & $v_5 - v_6$ \\
        6 & 6,7 & $v_6 - v_7$ \\
        7 & 7,8 & $v_7 - v_8$ \\
        8 & 8,9 & $v_8 - v_9$ \\
        9 & 9,10 & $v_9 - v_{10}$ \\
        10 & 10,11 & $v_{10} - v_{11}$ \\
        11 & 11,12 & $v_{11} - v_{12}$ \\
        12 & 12,13 & $v_{12} - v_{13}$ \\
        13 & 13,14 & $v_{13} - v_{14}$ \\
        14 & 14,15 & $v_{14} - v_{15}$ \\
        15 & 15,0 & $v_{15} - v_{0}$ \\
        
        \bottomrule
    \end{tblr}
    \label{table:measU}
\end{table}

按照上述方式将测得的电位数据映射为重建算法所需要的邻近测量模式的数据,并利用本文所提出的生成式EIT图像重建算法对其内部电导率分布进行重建,
即可获得重建结果。
此外,由于EIT采集系统所采用的采集帧率为每秒1帧,故本文将对每个扰动目标放入物理模型且稳定后静置1分钟,获得60帧数据。
随后在其中随机选择20帧,计算所选数据每个通道上的平均值,利用该值作为最终用于成像的电压数据,以此减小其由于水面波动以及其他外界因素导致的误差。

经上述步骤后,共生成50组可用于EIT图像重建的EIT电压数据,并将对应的电导率分布数据RGB图像作为判断该图像重建结果依据。由此获得50组电压-电导率分布(RGB图像)数据对。

随后,由于水槽中的电导率分布无法直接测量出精确值,故本采用数字图像处理相关技术,通过实验所得到的RGB扰动目标图像中背景帧和扰动目标像素的差异性,生成用于代表电导率分布的矩阵,如\cref{figure:PhyCond}所示。
\begin{figure}[H]
    \centering
    \includegraphics[width=.7\textwidth]{PhyCond.png}
    \caption{物理实验模型中电导率分布图像映射关系}
    \label{figure:PhyCond}
\end{figure}

\xsubsection{EIT物理实验数据分析与噪声生成器构建}{Analysis of EIT Physical Model Data and Construction of Noise Generator}
\label{NoiseRob}
由于EIT逆问题的病态性,传统的图像重建算法通常在物理模型数据中表现较差。而EIT正问题求解过程完备,且结果相对精确。故首先利用上一小节中生成的EIT电导率分布数据仿真出其边界电压分布,并将其与物理实验模型生成
的电压数据比较,如\cref{figure:expvol}所示。


\begin{figure}[h]
    \centering
    \includegraphics[width=.45\textwidth]{expvol.JPG}
    \caption{EIT物理模型实验与仿真结果的电压分布}
    \label{figure:expvol}
\end{figure}

由上述结果可知,EIT物理模型实验所得的电压分布相较于仿真结果有明显的噪声存在,
据分析可知,由于扰动目标的尺寸相对于水槽而言较小,这就导致可用信号的信息量变小。同时采集到的数据包含大量噪声,因此该采集结果中信噪比较低。

对于上述问题最直接的解决方案即通过物理模型得到不同电导率分布在场域内部不同位置的电压分布与电导率分布数据,从而利用这些物理模型数据训练EC-Diffusion 模型中的可学习参数。
然而,该方法的可行性较低:

首先,琼脂块的制作周期较长,并且琼脂块会与背景溶液发生离子交换,故物理模型会因为无法准确的设置电导率分布而产生误差。
其次,由于电压数据在测量的过程中难免会产生各式各样的误差(包括EIT的系统误差以及外界干扰所导致的误差),而实际任务中无法准确估计这类误差导致的噪声,故该方法仍然有可能导致重建质量差。
第三,由于琼脂块及易发生形变,故将多次设置在场域内部的不同位置实际操作难度大,并且可获得的数据量较小(由于每次测量需等待溶液稳定,由此也会产生一定量的误差),且数据集制作周期长。

并且由于物理模型实验所产生的噪声分布并不一定能通过包含噪声信息的电压数据作仿射变换得到(即,直接向原始数据中添加噪声训练的方式欠妥)。

故针对该类噪声,本文提出了一个噪声生成器,用于拟合该噪声分布。
该生成器将生成带噪声的电压分布,具体原理以及实现方案如下:

EIT物理模型实验中,背景帧的电压数据中包含噪声信号,而与之对应的仿真模型中的电压信号不包含噪声信号,故对应于特定电导率分布以及特定EIT物理模型的噪声分布可通过以上两个电压数据作非线性变换得到。
故利用一个可学习参数的神经网络作为噪声生成器,该生成器用于拟合上述非线性变换,其结构如\cref{figure:Generator}所示。

\begin{figure}[h]
    \centering
    \includegraphics[width=.9\textwidth]{Generator.png}
    \caption{噪声生成器架构}
    \label{figure:Generator}
\end{figure}


其中,该模型输入的$\mbs{v}_p$是EIT物理模型中不包含扰动目标的背景帧所对应的边界电压数据,$\mbs{v}_s$则为同样的背景帧在EIT仿真实验中所得到的边界电压数据。
 
首先该模型分别利用训练好的VEncoder分别对$\mbs{v}_p$和$\mbs{v}_s$进行编码,然后经过2个残差卷积层的特征提取,并利用CrossAttention层将编码后的$\mbs{v}_s$嵌入$\mbs{v}_p$中,最后再经过2个残差卷积层输出需要添加的噪声。
 
由于网络需要对噪声具有一定的抗干扰能力,故增加一项输出,该输出结果表示经过网络去噪后的电压分布。由于编码后的EIT数据维度相同(均为$32\times 32$),故这里可将输出层的通道数改为2。其中一个通道的输出代表原先的噪声隐变量,而新增加的另一个通道的输出则用来代表去噪后的电压分布。

该生成器与整个NoiseG共享损失,并共同训练,由此便可使该生成器生成适用于特定物理模型实验的噪声项。利用上述方式以及物理实验数据和仿真实验数据训练该模型,可得到网络的训练-测试Loss曲线如\cref{figure:LossGe}所示。


\begin{figure}[h]
    \centering
    \includegraphics[width=.6\textwidth]{LossGe.jpg}
    \caption{噪声生成器训练-测试Loss曲线}
    \label{figure:LossGe}
\end{figure}
可以看出,该网络能够很好的收敛,故验证了上述噪声生成器的可行性。


\xsection{算法性能分析}{Analysis of Performance of Algorithm}


\xsubsection{EIT物理模型实验图像重建结果分析}{Image Reconstruction Results of EIT Physical Model Experiment}
由于目前常见的基于深度学习的EIT图像重建算法大多利用判别式模型,且均使用到了卷积神经网络作为重建算法的主体,
故本文从中选择了一个2024年提出的基于CNN和RBFNN的模型\cite{RBFEIT}(下文中简称CNN-Based算法),与EC-Diffusion算法作对比,则
利用本文所提出的EC-Diffusion算法以及CNN-Based算法对EIT物理模型实验所采集到的电压数据进行重建,如\cref{figure:Reconsresexp}所示:


\begin{figure}[H]
    \centering
    \includegraphics[width=.9\textwidth]{Reconsresexp.PNG}
    \caption{EIT物理实验结果}
    \label{figure:Reconsresexp}
\end{figure}


该图为2个不同电导率分布(分别为0.05S/m和0.5S/m)的扰动目标分别在饱和硫酸钙溶液中3个不同位置下的电位向量(每个向量包含256个元素)的图像重建结果。
其中,左图是电导率为0.05S/m的扰动目标的重建结果,第一列为电导率分布的实况图,第二列为本文所提出的EC-Diffusion算法所得到的重建结果;第三列为CNN-Based算法的重建结果。

右图是电导率为0.5S/m的扰动目标的图像重建结果,其三列同上。


可以看出,EC-Diffusion算法可以重建出高质量的EIT物理模型实验的电导率分布图像,其重建质量明显优于CNN-Based图像重建算法。
并且该方法可有效区分不同电导率分布的扰动目标。同时该算法的时间效率高,满足脑卒中快速鉴别任务的时间性能要求。因此该算法可应用于脑卒中快速鉴别这一任务中。

此外,本文分别利用上述两种算法在仿真数据集上重建出了一组EIT电导率分布图像,如\cref{figure:reconsresult}所示。
\begin{figure}[H]
    \centering
    \includegraphics[width=.5\textwidth]{reconsresult.jpg}
    \caption{重建结果}
    \label{figure:reconsresult}
\end{figure}

其中第一列是原始的电导率分布图像,第二列是利用CNN重建后的结果,第三列则是本文所提出的EC-Diffusion重建算法所得到的结果。
可以看出,该算法在对图像背景电导率以及边缘电导率的重建效果均明显优于利用CNN重建的图像。

随后本文利用仿真数据以及\cref{geneD}中生成的物理实验电导率分布数据比较了上述俩种图像重建算法以及仅利用CNN实现的图像重建算法在不同数据集下重建结果的相对误差和相关系数,如\cref{table:EvaluationNice}(物理模型实验结果)和\cref{table:EvaSi}(仿真实验结果)所示。


\begin{table}[H]
  
    
    \caption{各模型最优参数对比(物理模型实验数据)}
    \begin{tblr}{
        colspec = {X[c] X[c] X[c]},
    }
    \toprule
    算法类型 & CC & RE($\times 10^{-5}$) \\
    \midrule
    EC-Diffusion & 0.8801 & 50.0943 \\
    CNN-Based & 0.7323 & 114.4946 \\
    \bottomrule
    \end{tblr}
    \label{table:EvaSi}
\end{table}


\begin{table}[H]
  
    
    \caption{各模型最优参数对比(仿真数据)}
    \begin{tblr}{
        colspec = {X[c] X[c] X[c]},
    }
    \toprule
    算法类型 & CC & RE($\times 10^{-5}$) \\
    \midrule
    EC-Diffusion & 0.9780 & 2.1073 \\
    CNN-Based & 0.9665 & 4.7563 \\
    \bottomrule
    \end{tblr}
    \label{table:EvaluationNice}
\end{table}
其中,EC-Diffusion为本文所提出的生成式EIT图像重建算法,CNN+RBFNN(Adam)和CNN(Adam)为\cite{RBFEIT}所提出的重建算法,
这里均已选择每种模型可能的最优参数来做比较。
可以看出,本文所提出的EC-Diffusion图像重建算法各项指标均高于目前先进的算法。


\xsubsection{EC-Diffusion算法对于噪声的鲁棒性分析}{Robustness Analysis of EC-Diffusion Algorithm for Noise}
根据上一小节的内容可以看出,本文所提出的EC-Diffusion算法在抗噪性上明显优于CNN-Based算法,故本节将利用仿真数据深入分析该算法的抗噪性能,进一步说明该算法对于脑卒中快速鉴别任务的先进性。


具体而言,通过向电压数据种加入随机噪声,使得其电压数据含有噪声分布,
随后利用图像重建算法重建出图像,进一步根据图像重建结果分析该算法的抗噪性。
该实验分别在信噪比为30、20、15、10对算法进行了验证,其中,带噪声的电压信号如\cref{figure:noiseV}所示。

\begin{figure}[h]
    \centering
    \includegraphics[width=0.8\textwidth]{noiseV.png}
    \caption{带噪声的电压分示意图}
    \label{figure:noiseV}
\end{figure}

利用上述电压和改进的EC-Diffusion算法,获得了以下重建结果,如\cref{fig:SNR}所示。


\newcommand{\subfigg}{
\begin{subfigure}[b]{0.8\linewidth}
\centering
\includegraphics[width=1\textwidth]{SNR30.png}
\subcaption{信噪比为30时的重建结果}
\end{subfigure}
}

\newcommand{\subfiggg}{
\begin{subfigure}[b]{0.8\linewidth}
\centering
\includegraphics[width=1\textwidth]{SNR20.png}
\subcaption{信噪比为20时的重建结果}
\end{subfigure}
}

\newcommand{\subfigggg}{
\begin{subfigure}[b]{0.8\linewidth}
\centering
\includegraphics[width=1\textwidth]{SNR15.png}
\subcaption{信噪比为15时的重建结果}
\end{subfigure}
}

\newcommand{\subfiggggg}{
\begin{subfigure}[b]{0.8\linewidth}
\centering
\includegraphics[width=1\textwidth]{SNR10.png}
\subcaption{信噪比为10时的重建结果}
\end{subfigure}
\label{subfigure:SNR30}
}

\begin{figure}[H]
    \subfigg
    \subfiggg
    %\floatcontinue{tb}
    \subfigggg
    \subfiggggg
    \caption{不同信噪比下的重建结果}
    \label{fig:SNR}
\end{figure}

可以看出,该算法在不同的信噪比下均可重建出高质量的EIT图像,并能够准确识别出扰动目标的电导率分布。
故EC-Diffusion算法对于噪声具有很强的鲁棒性,可有效应用于脑卒中快速鉴别的EIT图像重建任务中。

\xsubsection{编码器性能分析}{Analysis of Encoder}

本文所采用的EC-Diffusion模型中的VEncoder利用RBF-CNN结构实现,本节将首先分析RBF-CNN对于EIT电压数据的特征提取能力,从而说明RBF-CNN实现的VEncoder具有更强的特征提取能力。
随后对其时间、空间性能进行了分析,表明该模型能有效提高时间、空间利用率。
最后还对该编码器与EC-Diffusion模型中CEncoder所产生的编码结果之间的相关性进行分析,通过编码后的结果之间相关性的提高进一步验证了编码器-解码器的必要性。

\subsubsection{特征提取能力分析}

为验证该RBF-CNN对于EIT电压数据的特征提取能力,本节利用该网络实现一个EIT逆问题求解器。只要该网络能重建出明显高质量的EIT图像,则证明该网络可以提取出能表达EIT电压数据有关其对应的电导率分布的特征。

首先利用仿真模型获得EIT电压-电导率分布数据集,
其中,$d_m =  576$ ,即剖分的单元个数为576。$d_v = 192$ 即测量电压向量为192维。背景电导率为$0.15S/m$,扰动目标的中心电导率为0.7S/m。
大小为六边形,且其电导率根据距中心的距离而减小。


由于卷积层输出的每个元素都包含原始数据的空间特征,因此输出层包含 512 个神经元,代表更深层次的特征。
此后为一个径向基函数神经网络(RBFNN),输入层有512个神经元,输出层有576个神经元,分别代表每个元素的电导率。
众所周知,多层感知器包含点积(输入和权重之间)和激活(非线性)函数(例如 ReLU、Sigmoid 函数)。
网络训练通常是通过所有层的反向传播来完成的。

该网络首先包含 192 个神经元的输入层,分别对应于边界电压的每个维度。接下来有 6 个卷积层,每一步应用 3 × 3 卷积核和最大池化。最后两层应用 3×3 卷积核和 2×2 卷积核平均池,在每个卷积层之后使用指数线性单元(ELU)作为激活函数,因为 ELU 是。
由于卷积层输出的每个元素都包含原始数据的空间特征,因此输出层包含 512 个神经元,代表更深层次的特征。

利用训练数据初步重建模型,所得的结果如\cref{figure:recons_rbf}所示,其中,
第一行中的第一张图像是真实值。另外两个分别采用NR方法和学习率=0.0005模型的CNN + RBFNN(Adam)重建。
第二行的前两张图像由 CNN+ RBFNN (Adam) 重建,学习率 = 0.0001 和 0.001。第三个则则是由单独的CNN(Adam)重建。
\begin{figure}[h]
    \centering
    
    \includegraphics[width=.8\textwidth]{recons_rbf.png}
    
    \caption{重建结果对比}
    \label{figure:recons_rbf}
\end{figure}

\cref{table:RBFNNev}为对该网络性能的评估,其中RE和CC分别为相对误差(relative error)和 相关系数(correlation coefficient)。
(表中仅为最佳学习率下的性能)
\begin{table}[H]
    \centering
    \caption{网络评估}
    \label{table:RBFNNev}
    \begin{tblr}{colspec={X[c] X[c] X[c] X[c]}}
        \toprule
        模型类型 & 学习率 & CC & RE($\times 10^{-5}$) \\
        \midrule
        CNN+RBF(Adam) & 0.0005 & 0.9595 & 5.0552 \\ 
        CNN+RBF(SGDM) & 0.8& 0.9020 & 7.2939 \\
        CNN(Adam) & 0.0005 & 0.3694 & 15.0147 \\
        \bottomrule
    \end{tblr}
\end{table}

通过相对误差和重建结果可分析得出,带有RBF的重建结果相较于不带RBF的重建结果有显著提升,
因此认为RBFNN对于EIT电压-电导率映射的拟合有显著作用。
此外,根据实验结果可得出adam优化器对EIT图像重建任务具有显著的提升。

\subsubsection{时间性能分析}

由于基于深度学习的EIT图像重建算法在图像重建的过程中只需要执行一次网络的正向推理,因此该类算法相较于传统基于迭代的EIT图像重建算法有显著的时间性能优势。
由于两类算法的实现平台不同,以下比较具有一定的局限性,但其单次的计算时间具有多个数量级的差距,故仍可得出上述结论。

以32次重建为一组(与网络的batch数目相同)传统的N-R方法需要3.3941s(基于Matlab 2022实现, 操作系统环境:Windows 10)。
而利用本文所提出的EIT图像重建算法仅需要0.1877s(运算环境:python 3.10.12, Torch 2.2.1+cu121,GPU:Tesla 4 15360MiB,cuda 12.2, V12.2.140)。
故该算法具有明显的时间性能优势。

然而,即使EC-Diffusion算法相较于EIT传统算法已具有显著的时间性能优势,该算法的网络训练过程仍然非常耗时,其各模块训练时长共计约40小时。

\subsubsection{编码器所产生的中间结果分析}
除了通过图像重建效果反映其特征提取能力外,本节还对改进后的VEncoder以及第三章所提到的CEncoder得到的中间结果进行了分析。
具体而言,通过编码器的输入指出了该编码器-解码器结构对于模型重建效果的提升。
首先评估了该EIT重建任务中各个信号之间的相关性,如\cref{table:CoorSig}

\begin{table}[h]
  
    
    \caption{EIT信号之间的相关性分析}
    \begin{tblr}{
        colspec = {X[c] X[c] X[c]},
    }
    \toprule
    信号类型1 & 信号类型2 & 平均相关系数 \\
    \midrule
    原始的电压分布 & 原始的电导率分布 & 0.0049 \\
    编码后的电压分布 & 电导率分布 &  0.2724\\
    编码后的电压分布 & 编码后的电导率分布 & 0.5285 \\
    原始的电导率分布 & 编码后的电导率分布 & 0.5389 \\
    \bottomrule
    \end{tblr}
    \label{table:CoorSig}
\end{table}

可以看出,

 1)编码前的电压和电导率之间相关性极低(由于电压信号与电导率信号的维度不同,故采取给电压向量结尾补0的方式让其与电导率分布长度相同.)。
 2)经过编码后的电压信号与原始的电导率分布之间的相关性有明显提高,这将更有利于最终的图像重建任务。
 3)经过编码后的电导率信号和经过编码后的电压信号具有更强的相关性,同时经过编码后的电导率信号与原始的电导率同样具有较高的相关性,这表明编码器不但提高了电压-电导率之间的相关性,还不会破坏其与原始数据之间的关联。

更进一步,由于编码后的电压分布与电导率分布具有相同的特征数目,故可将编码后的电压信号利用其每个维度特征的值可视化,如\cref{figure:EncodedV1},\cref{figure:EncodedV2}所示。

\begin{figure}[h]
    \centering
    \includegraphics[width=0.8\textwidth]{EncodedV1.png}
    \caption{编码后的电压和真实电导率对比(大目标)}
    \label{figure:EncodedV1}
\end{figure}

\begin{figure}[H]
    \centering
    \includegraphics[width=0.8\textwidth]{EncodedV2.png}
    \caption{编码后的电压和真实电导率对比(小目标)}
    \label{figure:EncodedV2}
\end{figure}

其中,每组图像中,左侧为编码后的电压可视化后的结果,右侧为电导率分布的实际结果,可以看出,该编码器一定程度上让电压信号映射到了与电导率分布特征相关的隐空间中,
这种变换对于成像目标较大的电导率向量更敏感,而对于成像目标较小的数据效果不佳。
 





\xsection{小结}{Summary}


本章提出了一种用于生成EIT物理模型实验数据的策略,并利用之生成了一组用于模拟脑卒中患者的EIT数据,随后利用该数据以及EIT仿真数据分析了EC-Diffusion算法的性能,并与同类型先进算法作对比,证明了其有效性和先进性。
随后分析了该算法对于物理模型实验噪声的鲁棒性,并说明了EC-Diffusion模型中各个部分存在的必要性和有效性。
结果表明,本文所提出的生成式EIT图像重建算法
能有效地重建出高分辨率的EIT图像,能够应用于脑卒中快速鉴别任务中,为后续开展临床实验提供了理论基础
