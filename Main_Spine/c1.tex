% !TeX root = ../main.tex

\xchapter{绪论}{Introductions}

\xsection{研究背景和意义}{Backgrounds}

脑卒中是世界第二大致命的疾病,俗称为中风,除了高死亡率外,幸存者也可能会失去
视力、语言、瘫痪或神志不清。全世界每年有约 1500 万人患中风。其中 500 万人死亡,另
有 500 万人终身残疾,这严重危害到患者的健康,并且给家庭和社区造成了严重的负担。
因此,及时的治疗对于脑卒中的患者而言至关重要。通常而言,脑卒中由致病原因不同可分为两种类型,即
出血性脑卒中和缺血性脑卒中。缺血性脑卒中通常由于血栓阻塞了大脑部分的血液供应
而导致,出血性脑卒中则是由于脑中血管破裂引起的脑出血所导致。因此不同类型的脑
卒中所对应的治疗的手段有所不同。其中缺血性卒中通常可以利用溶栓剂(thrombolytic)治
疗;而出血性卒中则可通过血管内手术使其康复。同时,利用药物治疗缺血性脑卒中的有效
治疗时间通常在四个半小时内。由此可见,实现院前的脑卒中快速检测和类型诊断并展开救治,
能够降低脑卒中的致死致残率。

随着生物医学的飞速发展,各种先进的技术在医学成像中得到更加广泛应用。目前卒中
最主要的鉴别手段即是利用医学成像技术如 X-CT 所得到的结果进行判断。除此之外,医学
成像技术对于其他临床疾病的诊断和治疗有着巨大的意义,已成为临床诊断与医学研究中不
可或缺的一部分。计算机断层扫描(Computed Tomography, CT)作为过程层析成像
(Process Tomography, PT)的一种, 通过利用物理介质与人体发生相互作用来计算出其断层
图像,从而直观的呈现出人体组织器官的形态。目前国内外用于临床诊断的 CT 技术主包括
X 射线计算机断层扫描成像(X-ray computed tomography, X-CT)、超声层析成像(Ultrasound
Tomography)以及磁共振成像(Magnetic Resonance Imaging, MRI)等。在这些 CT 诊断技
术中,X-CT 的成像精度高,但是由于X射线的辐射对人体有害,因此在很多情况下不适于
大量重复使用,甚至不能对特定的群体使用(对于孕妇等)。超声CT利用超声波反射的信息
重建出人体内部的图像,是目前最安全的 CT 技术。但是由于难以建立超声波在人体内部吸
收、投射和反射的确定关系,目前的超声波成像技术忽略了超声波与人体组织之间的相互作
用而得不到理想的定量结果。MRI 可以提供分子水平的信息,能够发现人体生理化学的早期
变化,但是该技术成像速度较慢并且观察不到骨组织信息。
上述传统的 CT 诊断技术在临床上应用广泛,都能获得清晰的人体组织图像,但在医疗
实践中不尽如人意之处在于其成像成本高,价格昂贵、设备复杂并且对使用场景有一定的限
制,也不便于连续监测。对于卒中这类发病快、死亡率高的疾病而言,传统的 X-CT 由于设
备复杂庞大而无法实现及时的院前检测。因此人们迫切需要一种能够避免上述一部分缺陷从而
适用于院前脑卒中类型鉴别诊断的医学成像技术,
电阻抗成像(Electrical Impedance Tomography, EIT)技术就是其中之一。

EIT技术是一种无创、实时、可便捷化的新型医学成像技术。该技术通过向未知场内注入激励电流并测量场域边界的电压分布
从而重建出其内部电导率分布的图像。根据生物电化学理论可以知道,人体内不同组织的电导率(电阻率)不同(\cref{table:table1} 列出了人体主要组织和器官的阻抗分布,可以看到组织器官含水量和电导率
成正比),且同一组织不同健康状况下的电导率也不同(\cref{table:table2} 给出了大脑在不同病变状态下的阻抗变
化),这种差异性为 EIT 的技术的实现提供了生物医学基础。
对于脑卒中类型鉴别而言,由于缺血性脑卒中和出血性脑卒中所对应病灶区域的含水量有显著差异,因而患者颅内的阻抗分布应同样具有显著差异。
因此EIT能通过重构出脑卒中病灶组织的电导率从而区分其类型。

\begin{table}[H]
  \center
  \caption{20~100kHZ 下人体部分器官组织的阻抗分布}
  \label{table:table1}
  \begin{tblr}{X[c,2] X[c] X[c] X[c] X[c] X[c] X[c]}
    \toprule
    组织类型 & 脑髓 & 血液 & 肝脏 & 神经组织 & 脂肪 & 骨骼\\
    \midrule
    阻抗(Ω/cm) & 65 & 150 & 350	& 580	& 2060 & 16600 \\
    \bottomrule
  \end{tblr}
\end{table}
\begin{table}[H]
\center
\caption{4种情况下大脑阻抗变化范围}
\label{table:table2}
\begin{tblr}{X[c] X[c] X[c]}

  \toprule
  病变情况 & 数据来源 & 阻抗变化 \\
  \midrule
  脑肌肉萎缩 & Holder (1992) & 提高 70\% ~ 200\% \\
  扩展性 & Holder (1992)  & 提高 5\% ~ 25\% \\
  抑制癫痫 & Boone et al. (1994) & 增加或减少 7\% \\
  诱发反应 & Holder et al. (1996) & 减少 2\% ~ 5\% \\
  \bottomrule
\end{tblr}
\end{table}


相较于传统医学成像技术如X-CT、UR等,EIT技术具有以下优势:
1)EIT 是一种非侵入性、无损伤的监测技术。这是由于 EIT 利用低频、小电流形成的电
流场来获取人体内部阻抗分布信息,是安全而无害的。2)可以实现长时间连续的动态成像。
3)成像成本低、设备简单、使用方便,并且对使用场所的要求低,容易制成便携式设
  备。
4)属于功能性成像。EIT 不仅能反映组织的结构特性,更能获取人体生理、病理活动和
  健康状况的功能性医学信息。
基于上述优势,利用EIT技术实现对于卒中院前快速鉴别具有显著的可行性。
以往的研究\cite{8936983}\cite{2023Applied}\cite{Acosd}也表明利用 EIT 技术实
现脑卒中的快速监测是可行的。此外,由于 EIT 技术的这些优越性是其他传统 CT 所不具有
的,因此在许多的其他的医疗领域,如作为在髋关节置换手术中实时的可视化工具、血流中
凝块的成像、淋巴问题的非侵入性诊断\cite{2018MAET}
等都可以作为主要的成像手段或是传统 CT 技术的补充。特别是在新冠大流行期间,
Tomasino 等人利用 EIT 技术实时地评估了 COVID-19 呼吸衰竭患者旋前、旋中、旋后不同
肺区域的变化\cite{articleTSS},体现出 EIT 技术长期监测疾病的能力。除此之外,EIT 技术在其他领域也
得到了广泛的应用。2003 年,UC Berkeley 的 Kruger 等人探究了 EIT 在半导体制造过程中
提供时空信息来分辨晶圆状态信息的可行性\cite{2003Tomography}。2015 年 Carnegie Mellon 大学的 Y Zhang 和
Harrison 利用 EIT 技术实现了手势识别 \cite{CMU2015},并有望应用在可穿戴设备(如智能手表)上。
在化学工程中,EIT 广泛应用于气动输送、液体混合等。\cite{2003Chemical}此外,EIT 技术在材料科学\cite{2007CBEIT}、人
工敏感皮肤\cite{2020Artificial}等领域也得到了深入的研究。

纵然 EIT 打破了传统医学成像设备庞大、昂贵、对人体有害、对于场景要求高等不足,
其目前仍无法取代传统的医学成像技术。其一,EIT重建问题具有严重的病态性,即测量结果
微小的变化就可能导致重建质量大幅度地下降。其二,传统的 EIT 图像重建算法通常利用如
牛顿-拉弗逊(Newton-Raphson, N-R)等数值计算方法迭代求解出场域内部电导率(阻抗)分
布,在这个过程中会多次求解EIT正问题和逆问题,因此通常速度较慢。此外,由于颅
骨较低的电导率而导致进入颅内的电流强度大大减弱,从而使得采集到的信号质量差,这也给用于脑卒中鉴别任务的EIT图
像重建带来了一些麻烦。

近年来,深度学习模型因其对非线性映射强大的拟合能力而广泛应用在各个领域。主流
的深度学习模型如卷积神经网络(CNN)及其变体,也已被验证能够很好的提取高维数据的
复杂特征\cite{Yann2015Deep}。EIT 问题的关键就是建立从待测场边界的电压分布到场域内部电导率分
布的映射。同时,EIT 正问题结果相对准确,这就为基于深度学习的 EIT 图像重建
算法提供了大量数据支持。
根据对数据的建模方式不同,可将深度学习模型分为两类,分别为判别式模型和生成式
模型。然而,目前利用深度学习技术实现EIT图像重建算法的绝大部分研究都集中在判别式模型实现的重建算法上。
近年来,生成式模型在图像生成等任务重取得了显著的成果,由于生
成式模型更能建模数据本身的分布信息,故利用其拟合 EIT 图像重建问题的非线性映射将能
显著地提高EIT重建图像的质量。因此,基于生成式模型的EIT图像重建算法具有显著的研
究价值。

\xsection{国内外研究现状}{Relative Works}
\xsubsection{电阻抗断层成像及传统的图像重建算法}{EIT and Reconstruction Algorithm}

EIT 技术最早可以追溯到 20 世纪 30 年代,当时的地球物理学家利用一种类似 EIT 的技
术——电阻率层析成像技术(Electrical Resistivity Tomography, ERT) ,在地球表面钻孔放置
电极,或直接将电极放置在地球表面,向大地注入电流,并利用电极测得其表面的电压信号,以
此来确定测量区域内部的电导率分布,进一步进行相关的地质学研究(如结合已知岩面和矿物的
特性来探寻矿产资源的分布情况)\cite{1999Pottery}。1976 年,Wisconsin Madison 大学的
Swanson在他的博士论文中率先提出了 EIT 这一新颖的成像技术,引起了广泛的关注。
1978 年,Henderson和Webster设计并制造了一个“阻抗相机(impedance camera)”,
并利用其生成了人体胸部电阻抗图像,可以显示人体肺、心脏等位置,
但此时的图像还不是严格意义上的断层图像\cite{2007An}。
随后,1983 年,Sheffield 大学的 Barber等人利用等位线反投影算法重建出了人体前臂的断
层图像,其中低电导率区域可大体分辨出手臂骨骼。该算法通过假设边界电压和电阻率分布之间
的为线性关系,实现了快速的 EIT 图像重建。1987 年,Brown 等人建立了第一个完整的 EIT
数据测量系统\cite{1987Electrical}。该系统有 16 个电极,采用相邻电
极 5mA,50kHz 恒流激励。同年,Wisconsin Madison 大学的 Sakamoto 也建立了他们的 EIT
恒流激励测量系统。此后,由于 EIT 技术广阔的应用前景,吸引了越来越多的专家学者投入
到研究行列之中。

EIT 技术的关键一环就是其图像重建算法。近年来,越来越多的学者从不同的角度对 EIT
重建算法进行了改进。最初,根据成像目标的不同,可以将其分为两类:动态成像和静态成
像。如今,随着频率差分(Frequency-difference)成像的算法的提出,可将成像算法更细致的
分为三类:时间差分(Time- Difference)成像;绝对成像(静态成像);频率差分成像。其中,
时间差分成像,即最初被称为的动态成像算法,通过采集$t_0$和$t_1$两个不同时刻的边界电压数据
$v_0$和$v_1$,然后利用图像重建算法来计算出场域内部电导率分布的差值$\sigma_0-\sigma_1$
,从而重构出差分图像。以 Barber 等人提出的等位线反投影算法为代表。
由于测量数据相减时,其中的系统噪声会被消去,因此动态成像对于 EIT 系统的数据采集过程中
的系统误差有一定的抗干扰能力,因而实现起来较为容易,并且计算量较小。
而动态成像最大的缺点即由于成像需要前后时刻待测场内的阻抗发生变化,
因而无法对内部阻抗未发生变化的场进行成像,这就导致其应用范围窄。并且该算法假设电
流在同一平面内流动,故难以推广到三维 EIT 中。静态成像以待测场内的电导率分布绝对值
为计算目标,利用任何时间点测得的边界电压值直接绘制成电导率分布图像,而不需要其他
时间的数据,因此应用范围相较于动态成像更广。具有代表性的静态成像算法有美国
Wisconsin 大学 EIT 研究小组在其 EIT 成像系统中个应用的 Newton-Raphson(N-R) 算法、
1985 年 Muria 等人提出的敏感矩阵法等。频率差分成像算法也被称为多频
EIT(multifrequency EIT, MFEIT)。由于生物体组织的阻抗谱具有频率依赖性,因此 MFEIT 测量
不同频率下待测场内的阻抗分布,并利用它们重建了一组与组织特性相关的多拼电导率图
像。现有的 MFEIT 算法有基于单测量向量(SMV)行的单频段 EIT图像重建算法\cite{AreconsAlg}
\cite{7932511}\cite{2018Image}、基于多重测量向量(MMV)的 EIT 重建算法\cite{9732193}等。

\xsubsection{基于深度学习的EIT图像重建算法}{Deep Learning Based EIT Reconstruction Algorithm}

近年来,深度神经网络由于其对于非线性映射良好的拟合能力因而在模式识别和机器学
习领域得到了广泛的应用。EIT 图像重建问题本质上是利用待测场的边界电压-电流分布计
算出其内部的阻抗分布,即得到边界电压值到内部阻抗分布的非线性映射。EIT 逆问题的非
线性、病态性等特点是影响图像质量的主要原因。因此,利用深度神经网络强大的非线性映射拟合能力
来拟合EIT逆问题的映射,对于提高EIT重建图像分辨率具有显著的应用价值。
与此同时,训练好的深度学习模型的正向求解过程通常响应非常迅速,
因此利用深度学习技术实现的EIT图像重建算法通常具有较快的重建过程,
这也为EIT技术的更广泛应用提供了算法基础。同时也使得人工智能技术在医疗领域中得到了更广泛的应用。 

根据深度学习技术在重建算法中的作用可将基于深度学习的EIT图像重建算法分为三类,其分别为:
直接重建法、间接重建法和隐式重建法。本节将分别回顾每种算法的研究现状,并简单介绍其他领域中
利用深度学习技术实现的EIT图像重建算法的研究现状。

\subsubsection{直接重建法}

EIT图像重建问题就是要利用测得的电压分布计算出场域内部的电导率分布。
因此,可以利用深度神经网络对于非线性映射良好的拟合能力,直接建立EIT图像重建模型,
即向网络中输入电压分布(离散情况下为向量)进而获得电导率分布。
故将具有此类特征的算法称为直接重建法。

2016 年, Martin, Sebastien 等人提出了基于径向基函数(RBF)人工神经网网络的方法
来解决 EIT 重建问题,并证明了目标面积与 RBF 网络的扩散常数(spread constant)有很强
的相关性。该方法给出了更精确的小目标和大目标重建结果\cite{Ontheinf}。2017 年,S. Russo 等人利
用一个三层的人工神经元网络(ANN)实现了对 EIT 重建问题的求解,并首次将其应用在提
高基于 EIT 的人工敏感皮肤的触摸检测精度上。\cite{8233910} 2018 年,Wei Z 等人提出了一种基于迭
代的反演方法和一种基于卷积神经网络(CNN)的反演方法实现了 EIT 逆问题的求解,快速、
稳定、高质量地进行 EIT 成像。\cite{8606211} 2019 年,Shu-Wei Huang 等人利用 RBFNN 和 U-net 实
现了图像重建算法,该方法能有效的去除图像伪影(image artefact),并且可以更加准确的
重建出正确的目标位置\cite{8856781}。

此类算法通常直接利用神经网络拟合图像重建算法中的电压分布到电导率分布的映射,进而实现图像重建。
相较于传统重建算法而言,利用适当的数据集训练所得到的直接重建算法通常具有更高的重建分辨率。
并且由于已训练好的网络正向计算过程耗时短,因此也具有明显的时间性能优势。
然而,传统算法通常基于大量的数学物理模型演算而得,具有可靠的理论依据;
反之神经网络模型由于其黑箱的特征,故可解释性较差;并且数据集的构建对神经网络的性能影响极大。
因此,如何提高模型的可解释性并构建合理的数据集仍是该方法的一大难点。 

\subsubsection{间接重建法}
除了直接利用神经网络拟合EIT逆问题中边界电压-电流数据到电导率的非线性映射外,
部分学者还利用神经网络作为去噪网络,或作为整个重建过程的一部分,对EIT传统模型所得到的数值解进行后处理。
对于这种利用神经网络间接地重建出电导率分布的方法本文称其为间接重建法。 

2018 年,S. J. Hamilton 等人没有直接利用神经网络拟合 EIT 重建过程,而是利用 Dbar 方法实现了 EIT 图像重建,
并使用经仿真数据训练后的 CNN 进行图像后处理,有效的提高了重建质量\cite{8352045}。
同年,Martin S 等人在 EIT 问题求解器后添加全连接的人工神经元网
络(ANN)作为图像后处理模块,将重建的图像输入 ANN 以此来实现降噪的目的\cite{S2017A}。
2019 年,S. Ren 等人提出了一种两阶段深度学习(TSDL)方法,该方法通过一个预重建块重建
出低质量图像,然后结合形状数据利用 CNN 对重建结果进行后处理,能较好地重建肺形态,
其重建结果显著优于传统的全变差(TV)方法\cite{8907811}。同样利用 CNN 进行后处理的还有
Hamilton SJ 等人,他们将卷积神经网络与 D-bar Method 配对,设计了一种基于深度学习
的后处理算法,该方法可以很好的应对不确定的形状\cite{2019Beltrami}。 同年,Duan X 等人对 EIT 重建
后的图像进行后处理,利用卷积神经网络去除图像伪影,并将其应用在了人工皮肤 EIT 系统
中,显著提高图像的可视性并减少了重建图像与真实情况的误差\cite{2019Artificial}。
2020 年,N. Biasi 等人利用多层感知机实现了基于 EIT 触觉感知(tactile sensing)的逆
问题求解,其重建后的图像与仿真结果具有较高的一致性\cite{9278823}。同样利用多层感知机来实现
图像重建算法的还有 2020 年 Lee K 等人,他们利用一个全连接的多层感知拟合了从 EIT 电
压-电流数据到电导率的映射,实现了对腹部皮下脂肪的电导率分布的图像重建算法,用来
估计腹部皮下脂肪的厚度,实验结果表明这种厚度估计的方法能较好地应对脂肪均匀分布的
情况,并且对于其他区域具有很好的鲁棒性\cite{2020Electrical}。同年,Zhichao Lin 等人利用用神经网络的
监督下降法(SDMNN)实现了图像重建,该方案结合了神经网络收敛速度快和有监督下降法
(SDM)泛化能力强的优点,数值结果验证了该方法的有效性和准确性\cite{9060508}。Capps M 等人利
用卷积神经网络成功拟合了 EIT D-bar(一种非迭代式的重建算法) 方法中的散射变换与结
构内部边界之间的映射关系,并给出了仿真和实验数据的实例\cite{9130138}。由于神经网络的参数初
始化对于整个网络收敛具有一定的影响,同年 Zhou Chen 等人利用一个线性回归解初始化
FC-UNet 中的网络参数之后训练网络,并将训练好的网络应用在了微 EIT 传感器的细胞成像中 \cite{9128764}。

由于神经网络的可解释性较差,因此有学者也致力于在提高 EIT 重建质量的同时来保持
模型的可解释性。2020 年,S. Siltanen 等人演示了卷积神经网络如何代替最小二乘拟合,并
且在保持可解释性的同时,显著提高的模型精度\cite{9231717}。2022 年,Rong Fu 等人 提出了一种
将正则化重建方法的数学结构与深度学习相结合的正则化引导深度成像(RGDI)方法。实验证
明,该方法可以使图像重建仅仅依赖于测量数据而非人工建模或超参数的不同\cite{9739002}。
随着深度学习技术的进一步发展,部分学者开始尝试将前沿的技术应用在 EIT 问题求解
中。2021 年 Zainab Husain 等人利用一种基于低阶双变量多项式和 RBF 网络的电导率面分
解方法,有效地解决了 EIT 图像重建的逆问题。随后利用卷积网络和迁移学习对重建图像进
行分割,并利用 KNN 进行目标分类,获得了令人满意的分类精度\cite{9336698}。2022 年,Zhou Chen
等人构造了一种带有 Self-Attention 机制的长短期记忆网络(LSTM),很好的捕获了多频 EIT
频率内和频率间的依赖关系,并相对传统 MMV-ADMM 和最先进的深度学习方法而言有更
好的图像重建质量、收敛性能、噪声鲁棒性和计算效率\cite{9732193}。2022 年, Xinyu Zhang 等人提
出了一种由预成像块、特征提取块、图像重建块和图像去噪块四个部分组成的基于卷积神经
网络的 v 形去噪网络,重建效果明显优于 Tikhonov 正则化、卷积神经网络和 V-Net 方法重
建的图像\cite{9754540}。

此类方法通常将EIT图像重建任务分为了若干部分,利用深度学习方法或传统算法依次实现每一个部分,
进而实现完整的图像重建算法。相较于直接利用深度学习模型实现重建算法而言,
此类模型具有更高的可解释性,并通常由于模型设计的巧妙而具有更理想的成像分辨率。
然而部分EIT设备通常要求算法具有一定的实时性,模型的复杂程度增大通常会提高算法的时间和空间复杂度。
因此对于此类方法,在保持模型具有良好可解释性的优势的同时提高其时间和空间性能则成为未来研究的重点。

\subsubsection{隐式重建法}
基于EIT重构图像的医学征象鉴别和分类是当下的研究热点,如脑卒中类型的判别、生理和病理信息识别等,
但EIT图像的低分辨率是影响鉴别和分类准确性的重要因素。
因此,针对此类问题,部分研究利用机器学习,特别是深度学习技术尝试绕过生成电导率分布这一过程,
基于测得的电流-电压数据直接得到(重建出)分类结果或具体参数指标。
由于此类算法并不直接重建出场域内部的电导率分布,故将具有此类特征的重建算法称为隐式重建法。 

2020 年 McDermott B 等人利用 SVM 分类器对多频对称差分电阻抗断层成像(MFSDEIT)数据进行分类,以区分出血、血块以及正常情况,并在模拟数据和真人数据上达到了 85%
的平均精度\cite{article1234568}。2021年 Panji N. Darma等人利用KNN分割算法对肉类重构图像进行聚类,
成功实现了对肉类不同区域的分割\cite{9625686}。2021 年 Chiang S 等人利用 SVM 等传统机器学习
方法建立了几个不同的多分类器来分别进行活体猪组织分化的实验,结果表明,生物阻抗数
据可以有效地用于体内组织类型的鉴别\cite{app9194049}。2022 年 Nawapat Khumwa 等人建立了一个全
连接的多层感知机,拟合了了从阻抗分布到肺容积的映射,利用 EIT 中的生物电阻抗值来估
计肺容积,并在较高信噪比的情况下取得了理想的预测精度\cite{9741619}。2022 年, Yanyan Shi 等
人构建了一个多分类任务的 RCNN(残差卷积网络),将 EIT 测量数据作为网络的输入,并
输出分类结果。这一设计绕过了 EIT 逆问题求解的过程,因此避免了图像重建带来的误差,
并且在仿真数据和真实数据分别与全连接神经网络和 scnn 进行比较,证明了其分类结果优
于后两者\cite{9751762}。

此类方法将某项参数(如肺容积、代表某项疾病预测的布尔变量等)构成的空间看作隐空间,利用机器学习,
特别是深度学习算法,将EIT所测得的电压数据映射到隐空间中(通常为$\mathbb{R}^1$空间)从而获得该变量的值。
由于绕过了重建过程,因而模型通常情况下不需要对场域内部的电导率分布进行评估,
故而可以看作简单的分类(或聚类)器。但是这类方法通常由于设计目标单一而适用范围小。
因此对于部分特定任务可以采用此类算法。 

\subsubsection{其他领域中利用深度学习技术实现EIT图像重建算法的研究现状}

在工业 EIT 的研究中,Rymarczyk, T 等人在 2019 训练了在 EIT 和超声透射断层成像( UST)
背景下分别训练了多个逻辑回归子系统(LRS),每个 LRS 生成了重建图像的单个像素,以此
实现图像重建算法。该方法得益于其高效性因此可应用在工业界\cite{s19153400}。同年,他们将基于机
器学习的 EIT 重建算法与传统的确定性算法(Gauss-Newton 法)对比,结果表明 ANN 具有
最理想的成像效果,但其缺点是训练时间长,重建时间相对较长\cite{article123456}。

\xsubsection{生成式模型}{Generative Model}

根据深度学习模型所拟合的目标分布不同,可将其分为判别式模型和生成式模型两类。
其中,判别式模型如MLP、CNN等,目前已经广泛应用图像分类、图像分割以及其他领域的工程实践中。而生成式模型近年来才逐渐被人们重视起来。
生成式模型通常通过估计原始数据和观测变量的联合概率分布实现对于数据分布的学习。
传统的以朴素贝叶斯(Naive Bayes)高斯混合模型(GMMs)隐马尔可夫模型(hmm)线性判别分析 (LDA)为代表。

近年来主流的生成式模型通常包括Kingma等人提出的变分自编码器(Variational Auto Encoder)\cite{2013Auto},
Goodfellow等人于2014年提出的生成式对抗网络\cite{2014Generative}(Generative Adversarial Nets, GAN)
以及括2015年Jascha 等人提出的扩散模型\cite{DiffusionModel}(Diffusion model)。值得注意的是,
直到Jonathan Ho等人利用去噪扩散概率模型\cite{DDPM}(Denoising Diffusion Probabilistic Models,DDPM)生成出了高质量的图像以及之后,
以扩散概率模型为代表的生成式模型在图像生成、图像重建领域极强的潜力逐渐被人们重视起来。

2020年,Chen 等人利用改进的条件生成式对抗网络实现了图像重建过程。
其在鉴别器训练中加入了损失的判断,显著提高了训练的效率\cite{Chen2020}。
仿真结果表明,该方法能有效提高重建图像的清晰度和图像细节的重建能力。
2023年,Li 等人同样利用改进的CGAN实现了肺部EIT图像重建算法,实验结果表明,
该方法的重建图像病灶清晰、边界完整、位置准确,性能明显优于传统的基于CNN、U-net的方法\cite{2023SAR}。 


\xsection{本文研究内容及结构安排}{Main Content and Structure}

本文针对脑卒中快速鉴别这一具体任务,提出了生成式EIT图像重建算法,旨在提高EIT的重建效率和EIT图像重建分辨率。
针对本课题研究内容,本文共分为五个章节,其结构如下:

第一章为绪论。首简单介绍了脑卒中的特点、种类以及电阻抗断层成像技术以及其优势和不足,
并结合脑卒中快速鉴别任务给出了研究背景与意义。 随后简要概括了近年来EIT图像重建算法的发展历程,并指出其局限性,进而引出
近年来生成式模型的发展现状以及其在图像重建领域的显著成果。最后本文的研究内容和结构进行安排。

第二章主要阐述了EIT技术以及其图像重建算法的实现原理以及传统算法重建质量低、重建速度慢的弊端。
随后介绍了常见深度学习技术的原理和特点,最后引出了生成式模型的基本原理及其优势,为后续提出生成式EIT图像重建算法提供了理论基础。

第三章提出了生成式EIT图像重建算法。首先介绍了该算法的总体架构,即包括了数据编码部分,图像重建部分以及数据解码部分。
随后分别阐述了每个部分的设计原则以及其优势。最后利用精心设计的仿真数据对该模型的每个部分进行训练,并评估了各个部分以及算法整体的性能。
为后续开展用于脑卒中快速鉴别的EIT图像重建算法的物理模型实验提供了算法基础。

第四章首先针对脑卒中快速鉴别任务设计物理模型实验,并对该物理实验的各个环节做出了详细的阐述。
随后利用EIT采集系统采集了用于验证模型的数据,并对该数据进行分析,表明其与仿真电压分布的不同。
然后针对上述问题优化了上一章中提出的图像重建算法,并利用仿真数据对该优化后的网络进行训练和评估。
最后利用改进后的图像重建算法重建出了不同电导率分布的EIT图像,并对其重建结果做出了分析。
结果表明,本文所提出的用于脑卒中快速鉴别的生成式EIT图像重建算法
能有效区分不同电导率分布的扰动目标,进而指出该算法应用在脑卒中快速鉴别任务上的优势,为后续开展临床实验提供了坚实的算法保证。

第五章对本文所做的工作做出总结和概括,随后分析了当前工作存在的不足并给出了针对该问题的具体解决方案,最后对未来的研究做出展望。