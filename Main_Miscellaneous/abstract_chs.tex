
% !TeX root = ../main.tex
\begin{chineseabstract}

    脑卒中是当前全球第二大致死原因,通常分为缺血性卒中和失血性卒中,
    并对应了不同的治疗手段。
    因此对于卒中的快速,院前检测具有重要的研究价值。
    电阻抗断层成像(Electrical impedance Tomography, EIT)是一种新型的医学成像技术,具有便捷、实时、无辐射等优点。
    由于不同类型的卒中对应的颅内电导率分布具有显著差异,因此利用EIT实现对卒中类型的快速鉴别成为了可能。
    然而EIT图像重建算法的重建质量较低、重建时间较长等问题限制了其发展。
    而近年来,深度学习,尤其是生成式模型在图像生成、图像重建、图像超分辨等领域取得了显著的成果。
    本文将利用深度学习技术中的生成式模型实现EIT图像重建算法,并设计实验验证该算法对于卒中的快速鉴别的有效性。
    研究主要成果和创新点如下:

    1. 提出了利用生成式模型实现的EIT图像重建算法。由于传统的EIT图像重建算法的重建分辨率较低,并且重建时间较长,
    无法对EIT待测场内的目标进行快速、高质量的重建。针对该问题,本文开展了基于生成式模型的EIT图像重建算法的研究。
    该算法首先利用编码器对EIT数据进行编码:使得EIT编码后的数据能够适应本文提出的图像重建算法。
    随后利用改进的扩散模型为重建算法的主体,重建出编码后的图像;
    最后利用解码器对重建出的编码后的图像进行解码,进而重建出了高质量的EIT图像。
    仿真实验结果表名,该算法能够重建出高质量的EIT电导率分布图像,且能有效地区分出不同电导率分布的扰动目标在不同位置的情况。
    为后续将该算法应用物理模型实验提供了算法基础。
    

    2. 针对脑卒中快速鉴别任务,设计了物理实验并验证了算法的有效性。
    由于不同类型的卒中通常具有不同的电导率,本文利用琼脂块制作出具有不同大小、不同电导率的琼脂块,
    用来模拟不同类型的卒中。随后利用EIT采集系统以及设计好的背景场,分别采集了不同类型、
    不同大小的卒中的边界电压数据。随后验证了物理模型试验中电压分布与仿真实验中电压分布的不同,
    并根据其差异对本文提出的图像重建算法做出改进,最后利用优化后的算法重建出了不同类型卒中的电导率分布。
    实验结果表明,优化后的重建算法能有有效地区分出EIT物理模型不同扰动目标对应的的电导率分布。
    
    最终结果表明,本文所提出的用于脑卒中的生成式EIT图像重建算法可以重建出高质量的EIT电导率分布。为后续将生成式模型应用在EIT图像重建算法的研究打下基础。
    并且针对物理模型优化后的算法可以有效区分不同电导率分布的扰动目标,即可以区分不同类型的卒中病灶区域,为后续开展临床试验提供了技术基础。


%\footnotetext{*本研究得到某某基金(编号:)的资助}
% \astfootnote{本研究得到某某基金(编号:)的资助}
%\zhlipsum[2,5]

%本文做出了以下贡献:

%\begin{enumerate}[wide,]
    %\item \zhlipsum[1]
    %\item \zhlipsum[3]
%\end{enumerate}

\chinesekeywordstype{生成式模型;电阻抗断层成像;图像重建算法;深度学习}{应用研究}

\end{chineseabstract}

