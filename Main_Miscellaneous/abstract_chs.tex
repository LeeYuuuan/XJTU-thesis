
% !TeX root = ../main.tex
\begin{chineseabstract}

    脑卒中是当前全球第二大致死原因,通常分为缺血性卒中和失血性卒中,具有发病快、致死率高等特点。
    由于不同类型卒中所对应的治疗手段不同,因此对于卒中的快速,院前检测具有重要的研究价值。
    电阻抗断层成像(Electrical impedance Tomography, EIT)是一种新型的医学成像技术,
    通过向人体内注入安全的电流,并测量边界电压从而重建出人体内部电导率分布的图像。
    该技术由于其便捷、实时、无辐射等优点在临床实践中得到了广泛的应用。
    由于不同类型的卒中对应的颅内电导率分布具有显著差异,因此利用EIT实现对卒中类型的快速鉴别成为了可能。
    然而EIT图像重建算法的重建质量较低、重建时间较长等问题限制了其发展。
    而近年来,深度学习,尤其是生成式模型在图像生成、图像重建、图像超分辨等领域取得了显著的成果。
    本文将利用深度学习技术中的生成式模型实现EIT图像重建算法,并设计实验验证该算法对于卒中的快速鉴别的有效性。
    研究主要成果和创新点如下:

    1. 提出了利用生成式模型实现的EIT图像重建算法。由于传统的EIT图像重建算法的重建分辨率较低,并且重建时间较长,
    无法对EIT待测场内的目标进行快速、高质量的重建。针对该问题,本文开展了基于生成式模型的EIT图像重建算法的研究。
    该算法首先利用编码器对EIT数据进行编码;随后利用扩散模型为重建算法的主体,重建出编码后的图像;
    最后利用解码器对重建出的编码后的图像进行解码,进而重建出了高质量的EIT图像。此外,为了实现高效的EIT数据编码器,
    本文利用径向基函数神经网络设计了一个电压编码器,该部分显著提高了电压编码器的性能。
 
    2. 针对脑卒中快速鉴别任务,设计了物理实验并验证了算法的有效性。
    由于不同类型的卒中通常具有不同的电导率,本文利用琼脂块制作出具有不同大小、不同电导率的琼脂块,
    用来模拟不同类型的卒中。随后利用EIT采集系统以及设计好的背景场,分别采集了不同类型、
    不同大小的卒中的边界电压数据。最后利用上一节提出的算法重建出了不同类型卒中的电导率分布。
    此外,根据重建结果,本节还设计模型提高了采集到的电压信号质量,为重建出更高质量的EIT图像提供优质的测量数据。
    最终结果表明,本文所提出的用于脑卒中的生成式EIT图像重建算法可以有效区分不同电导率分布的扰动目标,进而为脑卒中快速鉴别任务提供了算法基础,
    为后续的临床试验提供了技术基础。

%\footnotetext{*本研究得到某某基金(编号:)的资助}
% \astfootnote{本研究得到某某基金(编号:)的资助}
%\zhlipsum[2,5]

%本文做出了以下贡献:

%\begin{enumerate}[wide,]
    %\item \zhlipsum[1]
    %\item \zhlipsum[3]
%\end{enumerate}

%\chinesekeywordstype{关键词1;关键词2;关键词3;Keywords;Test}{工程/项目管理}

\end{chineseabstract}

