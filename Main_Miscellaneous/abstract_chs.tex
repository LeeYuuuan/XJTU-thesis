
% !TeX root = ../main.tex
\begin{chineseabstract}

    脑卒中是当前全球第二大致死原因,通常分为缺血性脑卒中和出血性脑卒中,
    其分别对应不同的治疗手段。由于脑卒中的救治存在严格的时间窗口,
    因此对于卒中类型的快速、院前鉴别具有重要的研究价值。
    电阻抗断层成像(Electrical Impedance Tomography, EIT)是一种新型的医学成像技术,具有便捷、实时、无辐射等优点。
    由于不同类型的卒中对应的颅内电导率分布具有显著差异,因此利用EIT实现对卒中类型的快速鉴别具有可行性。
    然而由于EIT逆问题的病态性导致其图像重建算法的重建质量较低、重建时间较长,这进一步限制了其发展。
    近年来,深度学习技术,尤其是生成式模型在图像生成、图像重建、图像超分辨等领域取得了显著的成果。
    本文将利用深度生成式模型实现用于脑卒中快速鉴别的EIT图像重建算法,并设计了一种模拟脑卒中患者的EIT数据生成策略,验证了该算法对于脑卒中快速鉴别任务的有效性和先进性。
    研究主要的成果和创新点如下:

    提出了一个利用生成式模型实现的EIT图像重建算法。传统EIT成像算法的重建图像分辨率较低,重建时间较长。
    针对该问题,本文开展了生成式EIT图像重建算法的研究。
    设计了一个用于脑卒中快速鉴别的EIT图像重建算法。该算法的主体由一个多模块的深度学习模型实现。该模型共包含一个编码器、一个解码器以及一个基于改进扩散模型的图像重建块。
    随后设计仿真实验训练并评估了该模型。
    仿真实验结果表明,该算法能够重建出高质量的EIT电导率分布图像,且能有效地区分出不同电导率分布的扰动目标在不同位置的情况。
    为后续将该算法应用于真实的EIT物理模型上提供了算法基础。
    

    针对脑卒中快速鉴别任务,设计了物理模型实验并提出了一种数据生成策略,随后利用物理模型实验数据以及仿真数据分析了该算法整体的性能以及各模块的性能,验证了其在真实物理模型上的有效性和先进性。
    由于不同类型的脑卒中病灶区域具有不同的电导率,故本文设计了一种数据生成策略,采用琼脂材料制作具有不同大小、不同电导率的琼脂块,分别用于模拟出血和缺血性脑卒中的病灶区域。
    随后设计了EIT物理模型实验并利用该实验采集并生成了模拟脑卒中患者的EIT数据。
    最后利用相关数据对本文所提出的EIT图像重建算法进行了分析与评估,并对其各个模块的必要性和先进性作出说明。
    实验结果表明,本文所提出的EIT图像重建算法能重建出高质量的EIT图像,并有效地区分出真实EIT物理模型中不同电导率的扰动目标,可用于脑卒中的快速鉴别任务中。
    
    最终结果表明,本文所提出的用于脑卒中快速鉴别的生成式EIT图像重建算法可以重建出高质量的EIT电导率分布,即可以区分不同类型卒中的病灶区域。为后续开展相关临床实验提供了技术基础,
    并为后续将生成式模型应用在EIT图像重建算法的研究打下基础。


%\footnotetext{*本研究得到某某基金(编号:)的资助}
% \astfootnote{本研究得到某某基金(编号:)的资助}
%\zhlipsum[2,5]

%本文做出了以下贡献:

%\begin{enumerate}[wide,]
    %\item \zhlipsum[1]
    %\item \zhlipsum[3]
%\end{enumerate}

\chinesekeywordstype{生成式模型;电阻抗断层成像;图像重建算法;深度学习}{应用研究}

\end{chineseabstract}

